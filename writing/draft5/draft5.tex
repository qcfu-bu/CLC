\documentclass[sigplan,screen]{acmart}

\usepackage[utf8]{inputenc}

% packages
\usepackage{graphicx}
\usepackage{amsthm}
\usepackage{amsmath}
\usepackage{mdwtab}
\usepackage{syntax}
\usepackage{stmaryrd}
\usepackage{mathtools}
\usepackage{mathpartir}
\usepackage{listings}
\usepackage{float}
\usepackage{tikz-cd}
\usepackage{enumitem}
\usepackage{minted}

\setlist[itemize]{leftmargin=1.6em}
\renewcommand{\syntleft}{}
\renewcommand{\syntright}{}
\setlength{\grammarparsep}{7pt}
\setlength{\grammarindent}{7em}
\newcommand{\indalt}[1][2]{\\\hspace*{-1.2em}\textbar\quad}
\newcommand{\rname}[1]{\textsc{\footnotesize #1}}
\newcommand{\pure}[1]{|#1|}
\newcommand{\refl}{\text{refl}}
\newcommand{\letin}[3]{\ $\text{let }#1\text{ := }#2\text{ in }#3$\ }
\newcommand{\ind}[1]{\text{Ind}_{#1}}
\newcommand{\constr}{\text{Constr}}
\newcommand{\case}{\text{Case}}
\newcommand{\dcase}{\text{DCase}}
\newcommand{\fix}{\text{Fix }}
\newcommand{\sep}{\text{ | }}
\newcommand{\unit}{\text{unit}}
\newcommand{\new}{\text{new}}
\newcommand{\free}{\text{free}}
\newcommand{\get}{\text{get}}
\newcommand{\set}{\text{set}}
\newcommand{\session}{\text{session}}
\newcommand{\channel}{\text{channel}}
\newcommand{\open}{\text{open}}
\newcommand{\close}{\text{close}}
\newcommand{\send}{\text{send}}
\newcommand{\recv}{\text{recv}}
\newcommand{\SEND}{\texttt{SEND}}
\newcommand{\RECV}{\texttt{RECV}}
\newcommand{\END}{\texttt{END}}
\newcommand{\utype}{:_{\scriptscriptstyle U}}
\newcommand{\ltype}{:_{\scriptscriptstyle L}}
\newcommand{\stype}[1]{:_#1}
\newcommand{\step}{\leadsto}
\newcommand{\pstep}{\leadsto}
\newcommand{\mrg}[3]{#1\ddagger#2\ddagger#3}
\newcommand{\erase}[1]{\llbracket #1 \rrbracket}
\newcommand{\lift}[1]{\llparenthesis #1 \rrparenthesis}
\newcommand{\lrangle}[1]{\langle #1 \rangle}
\newcommand{\ucons}{constructor_{\scriptscriptstyle U}}
\newcommand{\lcons}{constructor_{\scriptscriptstyle L}}
\newcommand{\scons}{constructor_{s}}
\newcommand{\inl}{\text{inl}}
\newcommand{\inr}{\text{inr}}

\title{The Calculus of Linear Constructions}
\author{Qiancheng Fu}
\affiliation{
  \institution{Boston University}
  \city{Boston}
  \state{MA}
  \country{USA}
}
\email{qcfu@bu.edu}

\keywords{Dependent types, linear types, inductive types, Calculus of Constructions}

\begin{document}
\begin{abstract}
  The Calculus of Linear Constructions (CLC) is an extension of the Calculus of Constructions (CC) with linear types. Specifically, CLC extends the predicative CC$\omega$ with a hierarchy of linear universes that precisely controls the weakening and contraction of its term level inhabitants. We study the meta-theory of CLC, showing that it is a sound logical framework for reasoning about resource. We further extend CLC with rules for defining inductive linear types in the style of CIC, forming the Calculus of Inductive Linear Constructions (CILC). Through examples, we demonstrate that CILC facilitates correct by construction imperative programming and lightweight protocol enforcement. We have formalized and proven correct all major results in the Coq Proof Assistant.
\end{abstract}
\maketitle


\section{Introduction}
The Calculus of Constructions (CC) is a dependent type theory introduced by Coquand and Huet in their landmark work \cite{cc}. In CC types can depend on terms, allowing one to write precise specifications as types. Today, CC and its extensions CIC \cite{cic} and ECC \cite{ecc} lie at the core of popular proof assistants such as Coq \cite{coq}, Agda \cite{agda}, Lean \cite{lean} and others. These theorem provers have found great success in the fields of software verification, and constructive mathematics.

However, due to its origins as a logical framework for constructive mathematics, it is quite difficult for CC to encode and reason about resources. Intuitively, a mathematical theorem can be applied an unrestricted number of times. Comparatively, the usage of resources is more limited. For example, if we encode Girard's classical example \cite{girard95} of purchasing cigarettes literally into CC as a function of type:
\begin{equation*}
  Money \rightarrow Camels + Marlboro
\end{equation*}
If viewed as a propositional implication, the customer will still maintain full ownership of their money after paying the vendor, because implication does not diminish the validity of its antecedent. Unless the vendor is exceedingly generous, we are faced with the crime of counterfeiting. Users of proof assistants based on CC often need to embed external logics such as Separation Logic to provide domain-specific reasoning principles for dealing with resource. We propose an alternative solution: extend CC with linear types.

This paper presents a new linear dependent type system --- The Calculus of Linear Constructions (CLC). CLC extends the predicative CC$\omega$ with linear types. CC$\omega$ itself is an extension of CC with a cumulative hierarchy of type universes. We add an extra universe sort $L$ of linear types with cumulativity parallel to the sort $U$ of non-linear types. This ultimately cumulates in the \textit{linearity} theorem, stating that all resources are used exactly once.

The presence of both linear and dependent types enables CLC to write specifications that faithfully encode the usage of resources. The previous example of monetary transaction can be refined using an indexed linear type family $Money : \mathbb{N} \rightarrow L$ as follows.
\begin{equation*}
  (x \ltype Money\ 5) \rightarrow (Camels \oplus Marlboro)
\end{equation*}
This new specification for the transaction demands a payment of 5 units of money, and the customer be relieved of their ownership after the transaction finishes, effectively preventing the contradiction of having your cake and eating it too.

Compared to preexisting approaches for integrating linear types and dependent types, CLC offers a ``lightweight'' approach, akin to Mazurak et al.'s work on System F \cite{mazurak}. Prior works have either utilized an explicit ! exponential or a pair of adjoint connectives to mediate between the worlds of linear and non-linear types. Our method requires neither exponential nor adjoint connectives explicitly in the syntax, allowing for a straightforward modeling of CLC in the predicative CC$\omega$, and lifting of predicative CC$\omega$ into CLC with minimal annotations needed.

We further extend CLC with rules for defining inductive linear types in the style of CIC, forming the Calculus of Inductive Linear Constructions (CILC). Linear connectives such a $\otimes$ and $\oplus$ are definable using this generalized mechanism in a straightforward way. Through examples, we show that CILC facilitates lightweight reasoning about stateful programs and protocol enforcement.

We have formalized all major results in Coq and implemented a prototype in OCaml, both of which are available in the first author's Github repository.

\noindent \textbf{\textit{Contributions}}:
Our contributions can be summarized as follows.
\begin{itemize}
  \item Fist, we describe the Calculus of Linear Constructions, an extension to the Calculus of Constructions with linear types. The integration of linear types and dependent types allows CLC to directly and precisely reason about resource.
  \item Second, we study the meta-theory of CLC directly, showing that it satisfies the standard properties of confluence, validity and subject reduction.
  \item Next, we prove the linearity theorem, showing the tangible impact linear types have on the structure of terms. We also show that promotion and dereliction of linear logic are derivable as theorems for canonical types.
  \item Additionally, we observe that CLC is backwards compatible with the predicative CC$\omega$, allowing us to construct a reduction preserving model of CLC in CC$\omega$, showing that CLC is consistent. We prove that all valid CC$\omega$ terms are valid CLC terms after adding minimal amounts of annotation.
  \item Furthermore, we extend CLC with rules to define inductive linear types in the style of CIC, forming the Calculus of Inductive Linear Constructions (CILC). This is the first presentation of a generalized mechanism for defining inductive linear types that we are aware of.
  \item All major results have been formalized and proven correct in the Coq Proof Assistant with help from the Autosubst \cite{autosubst} library. To the best of our knowledge, our development is the first machine checked formalization of a linear dependent type theory.
  \item Finally, we give an implementation of CILC in OCaml. Algorithmic type checking employed by the typechecker streamlines the process of writing CILC programs.
\end{itemize}

\section{The Language of CLC}
\subsection{Syntax}
\begin{figure}[h]
  \caption{Syntax of CLC}
  \begin{grammar}
    <$i$> := 0 | 1 | 2 ... \phantom{* |} \hspace*{3em} universe levels

    <$s, t$> ::= $U$ | $L$ \phantom{| $x$} \hspace*{4.6em} sorts

    <$m, n, A, B, M$> ::= $U_i$ | $L_i$ | $x$ \hspace*{4.2em} expressions
    \indalt $(x :_s A) \rightarrow B$
    \indalt $(x :_s A) \multimap B$
    \indalt $\lambda x :_s A. n$
    \indalt $m\ n$
  \end{grammar}
  \Description{}
  \label{syntax1}
\end{figure}

The syntax of the core type theory is presented in Figure \ref{syntax1}. Our type theory contains two sorts of universes $U$ and $L$. We use the meta variable $i$ to quantify over universe levels $0, 1, 2, ...$. $U_i$, and $L_i$ are the universes at level $i$ of non-linear and linear types respectively.
\begin{figure}[h]
  \caption{Correspondence of CLC types and MELL implications}
  \Description{}
  \begin{align}
    (\_ \utype A) \rightarrow B \quad & \Leftrightarrow \quad !(!A \multimap B) \\
    (\_ \ltype A) \rightarrow B \quad & \Leftrightarrow \quad !(A \multimap B)  \\
    (\_ \utype A) \multimap B \quad   & \Leftrightarrow \quad !A \multimap B    \\
    (\_ \ltype A) \multimap B \quad   & \Leftrightarrow \quad A \multimap B
  \end{align}
  \label{correspondence}
\end{figure}

A clear departure of our language from standard presentations of both linear type theory and dependent type theory is the presence of two function types: $(x :_s A) \rightarrow B$, $(x :_s A) \multimap B$. The reason for these function types is that we have built the ! exponential of linear logic implicitly into universe sorts. The sort annotation $s$ here records the universe sort of the function domain. The behavior of ! is difficult to account for even in simple linear type theories. Subtle issues arise if !! is not canonically isomorphic to !, which may invalidate the substitution lemma \cite{substitute}. By integrating the exponential directly into universe sorts, we have limited ! to only be canonically usable. This allows us to derive the substitution lemma, and construct a direct modeling of CLC in CC$\omega$ without requiring any additional machinery for manipulating exponential.

Figure \ref{correspondence} illustrates the correspondence between CLC functions and Multiplicative Exponential Linear Logic (MELL) implications. MELL lacks counterparts for the cases (1), (3) if the co-domain $B$ is dependent on arguments of domain $A$. We will discuss function type formation in Section \ref{tyformation} in greater detail.

Algorithmic type checking techniques allow users to omit writing sort indices for many situations in practice. Our implementation employs bi-directional type checking and users rarely interact with sort indices in the surface syntax.

\subsection{Universes and Cumulativity}
CLC features two sorts of universes $U$ and $L$ with level indices $0, 1, 2, \cdots$. $U_i$ and $L_i$ are the predicative universes of non-linear types and linear types respectively. The main mechanism that CLC uses to distinguish between linear and non-linear types is by checking the universe to which they belong. Basically, terms with types that occur within $U_i$ are unrestricted in their usage. Terms with types that occur within $L_i$ are restricted to being used exactly once.

In order to lift terms from lower universes to higher ones, there exists cumulativity between universe levels of the same sort. We define cumulativity as follows.

\begin{definition}
  The cumulativity relation ($\preceq$) is the smallest binary relation over terms such that
  \begin{enumerate}
    \item $\preceq$ is a partial order with respect to equality.
          \begin{enumerate}
            \item If $A \equiv B$, then $A \preceq B$.
            \item If $A \preceq B$ and $B \preceq A$, then $A \equiv B$.
            \item If $A \preceq B$ and $B \preceq C$, then $A \preceq B$.
          \end{enumerate}
    \item $U_0 \preceq U_1 \preceq U_2 \preceq \cdots$
    \item $L_0 \preceq L_1 \preceq L_2 \preceq \cdots$
    \item If $A_1 \equiv A_2$ and $B_1 \preceq B_2$, \\ then
          $(x \stype{s} A_1) \rightarrow B_1 \preceq (x \stype{s} A_2) \rightarrow B_2$
    \item If $A_1 \equiv A_2$ and $B_1 \preceq B_2$, \\ then
          $(x \stype{s} A_1) \multimap B_1 \preceq (x \stype{s} A_2) \multimap B_2$
  \end{enumerate}
\end{definition}

Figure \ref{universe} illustrates the structure of our universe hierarchy. Each linear universe $L_i$ has $U_{i+1}$ as its type, allowing functions to freely quantify over linear \textit{types}. However, $L_i$ cumulates to $L_{i+1}$. These two parallel threads of cumulativity prevent linear types from being transported to the non-linear universes, and subsequently losing track of linearity.

\begin{figure}[h]
  \caption{The Universe Hierarchy}
  \vspace{1em}
  \centering
  \begin{tikzcd}
    U_0 \arrow[r, ":"', "\preceq", dash]
    & U_1 \arrow[r, ":"', "\preceq", dash]
    & U_2 \arrow[r, ":"', "\preceq", dash]
    & \cdots \\
    L_0 \arrow[r, "\preceq" description, dash] \arrow[ru, ":" description, dash]
    & L_1 \arrow[r, "\preceq" description, dash] \arrow[ru, ":" description, dash]
    & L_2 \arrow[r, "\preceq" description, dash] \arrow[ru, ":" description, dash]
    & \cdots
  \end{tikzcd}
  \label{universe}
  \Description{}
\end{figure}

\subsection{Context and Structural Judgments}
The context of our language employs a mixed linear/non-linear representation in the style of Luo\cite{luo}. Variables in the context are annotated to indicate whether they are linear or non-linear. A non-linear variable is annotated as $\Gamma, x \utype A$, whereas a linear variable is annotated as $\Gamma, x \ltype A$.

Next, we define a $\mrg{\Gamma_1}{\Gamma_2}{\Gamma}$ relation that merges two mixed contexts $\Gamma_1$, and $\Gamma_2$ into $\Gamma$, by performing contraction on shared non-linear variables. For linear variables, the $\mrg{\_}{\_}{\_}$ relation is defined if and only if each variable occurs uniquely in one context and not the other. This definition of $\mrg{\_}{\_}{\_}$ is what allows contraction for unrestricted variables whilst forbidding it for restricted ones.

An auxiliary judgment $\pure{\Gamma}$ is defined to assert that a context $\Gamma$ does not contain linear variables. In other words, all variables found in $\pure{\Gamma}$ are annotated of the form $x \utype A$. The full rules for structural judgments are presented in Figure \ref{structural}.

\begin{figure}[h]
  \caption{Structural Judgments}
  \begin{mathpar}
    \inferrule
    { }
    { \epsilon \vdash }
    \rname{Wf-$\epsilon$}

    \inferrule
    { \Gamma\ \vdash \\
      \overline{\Gamma} \vdash A : U_i }
    { \Gamma, x \utype A \vdash }
    \rname{Wf-U}

    \inferrule
    { \Gamma\ \vdash \\
      \overline{\Gamma} \vdash A : L_i }
    { \Gamma, x \ltype A\ \vdash }
    \rname{Wf-L}

    \inferrule
    { }
    { \pure{\epsilon} }
    \rname{Pure-$\epsilon$}

    \inferrule
    { \pure{\Gamma} \\
      \Gamma \vdash A : U_i }
    { \pure{\Gamma, x \utype A} }
    \rname{Pure-U}

    \inferrule
    { }
    { \mrg{\epsilon}{\epsilon}{\epsilon} }
    \rname{Merge-$\epsilon$}

    \inferrule
    { \mrg{\Gamma_1}{\Gamma_2}{\Gamma} }
    { \mrg{\Gamma_1, x \utype A}
      {\Gamma_2, x \utype A}
      {\Gamma, x \utype A} }
    \rname{Merge-U}

    \inferrule
    { \mrg{\Gamma_1}{\Gamma_2}{\Gamma} \\
      x \notin \Gamma_2 }
    { \mrg{\Gamma_1, x \ltype A}
      {\Gamma_2}
      {\Gamma, x \ltype A} }
    \rname{Merge-L1}

    \inferrule
    { \mrg{\Gamma_1}{\Gamma_2}{\Gamma} \\
      x \notin \Gamma_1 }
    { \mrg{\Gamma_1}
      {\Gamma_2, x \ltype A}
      {\Gamma, x \ltype A} }
    \rname{Merge-L2}
  \end{mathpar}
  \label{structural}
  \Description{}
\end{figure}

\begin{definition}
  The context restriction function $\overline{\Gamma}$ is defined as a recursive filter over $\Gamma$ as follows. All linear variables are removed from context $\Gamma$. The result of context restriction is the non-linear subset of the original context.
  \begin{align*}
    \overline{\epsilon} = \epsilon
    \hspace*{4em}
    \overline{\Gamma, x \utype A} = \overline{\Gamma}, x \utype A
    \hspace*{4em}
    \overline{\Gamma, x \ltype A} = \overline{\Gamma}
  \end{align*}
\end{definition}

\subsection{Typing Judgment}
Typing judgments in CLC take on the form of $\Gamma \vdash m : A$. Intuitively, this judgment states that the term $m$ is an inhabitant of type $A$, with free variables typed in $\Gamma$.

\begin{definition} We formally define the terminology \textit{non-linear}, \textit{linear}, \textit{unrestricted} and \textit{restricted}.
  \begin{enumerate}
    \item A \textit{type} $A$ is \textit{non-linear} under context $\Gamma$ if $\Gamma \vdash A : U_i$.
    \item A \textit{type} $A$ is \textit{linear} under context $\Gamma$ if $\Gamma \vdash A : L_k$.
    \item A \textit{term} $m$ is \textit{unrestricted} under context $\Gamma$ if there exists non-linear type $A$, such that $\Gamma \vdash m : A$. A unrestricted term may be used an arbitrary number of times.
    \item A \textit{term} $m$ is \textit{restricted} under context $\Gamma$ if there exists linear type $A$, such that $\Gamma \vdash m : A$. A restricted term must be used exactly once.
  \end{enumerate}
\end{definition}

\subsection{Type Formation} \label{tyformation}
The rules for forming types are presented in Figure \ref{type}. In CLC, we forbid types from depending on linear terms similar to \cite{llf,neel15} for the same reason of avoiding philosophical troubles.

The axiom rules \rname{U-Axiom}, \rname{L-Axiom} are almost standard, the main difference being the extra side-condition of judgment $\pure{\Gamma}$. In most presentations of dependent type theories without linear types, the universe axioms are derivable under any well-formed context $\Gamma$. Variables not pertaining to actual proofs could be introduced this way, thus giving rise to the admissibility of weakening. To support linear types, we must restrict weakening to non-linear variables. This justifies the restriction of $\Gamma$ to contain only non-linear variables for \rname{U-Axiom}, \rname{L-Axiom}. From \rname{L-Axiom} we can see that the universe of linear types $L_i$ is an inhabitant of $U_{i+1}$. This is inspired by Krishnaswami et al.'s treatment of linear universes \cite{neel15}, where linear \textit{types} themselves can be used unrestrictedly.

\begin{figure}[h]
  \caption{Type Formation}
  \begin{mathpar}
    \inferrule
    { \pure{\Gamma} }
    { \Gamma \vdash s_i : U_{i+1} }
    \rname{Sort-Axiom}

    \inferrule
    { \pure{\Gamma} \\
      \Gamma \vdash A : U_i \\
      \Gamma, x \utype A \vdash B : s_i }
    { \Gamma \vdash (x \utype A) \rightarrow B : U_i }
    \rname{U$\rightarrow$}

    \inferrule
    { \pure{\Gamma} \\
      \Gamma \vdash A : L_i \\
      \Gamma \vdash B : s_i \\
      x \notin \Gamma }
    { \Gamma \vdash (x \ltype A) \rightarrow B : U_i }
    \rname{L$\rightarrow$}

    \inferrule
    { \pure{\Gamma} \\
      \Gamma \vdash A : U_i \\
      \Gamma, x \utype A \vdash B : s_i }
    { \Gamma \vdash (x \utype A) \multimap B : L_i }
    \rname{U$\multimap$}

    \inferrule
    { \pure{\Gamma} \\
      \Gamma \vdash A : L_i \\
      \Gamma \vdash B : s_i \\
      x \notin \Gamma }
    { \Gamma \vdash (x \ltype A) \multimap B : L_i }
    \rname{L$\multimap$}
  \end{mathpar}
  \label{type}
  \Description{}
\end{figure}

The \rname{U$\rightarrow$} rule is used for forming non-linear function types with non-linear domains. This is evident from the judgment $\Gamma \vdash A : U_i$. Due to the fact that $A$ is in the non-linear universe $U_i$, terms of type $A$ have unrestricted usage. The non-linearity of domain $A$ allows $B$ to depend on terms of type $A$, as seen in judgment $\Gamma, x \utype A \vdash B : s_i$. The domain $B$ itself may be non-linear or linear, since $B$'s universe $s_k$ can vary between $U_i$ and $L_i$. From the resulting judgment $\Gamma \vdash (x \utype A) \rightarrow B : U_i$, we see that the overall type is a non-linear type. The term level $\lambda$-abstractions of these types can be used unrestrictedly.

The \rname{L$\rightarrow$} rule is used for forming non-linear function types with linear domains. From the judgment $\Gamma \vdash A : L_i$, we see that $A$ is a linear type, and terms of type $A$ have restricted usage. Because of this, we forbid co-domain $B$ from depending on terms of type $A$, evident in the judgment $\Gamma \vdash B : s_i$. Like \rname{U$\rightarrow$}, co-domain $B$ itself may be non-linear or linear, since $B$'s universe $s_i$ can vary between $U_i$ and $L_i$. The final resulting judgment is $\Gamma \vdash (x \ltype A) \rightarrow B : U_i$. Here, $x$ is a hypocritical unbinding variable whose only purpose is to preserve syntax uniformity. Again, the overall resulting type is non-linear, so the term level $\lambda$-abstractions of these types can be used unrestrictedly.

The \rname{U$\multimap$}, \rname{L$\multimap$} rules are similar to \rname{U$\rightarrow$}, \rname{L$\rightarrow$} in spirit. The dependency considerations for domain $A$ and co-domain $B$ are exactly the same. The main difference between \rname{U$\multimap$}, \rname{L$\multimap$} and \rname{U$\rightarrow$}, \rname{L$\rightarrow$} is the universe sort of the resulting type. The sorts of the function types formed by \rname{U$\multimap$}, \rname{L$\multimap$} are $L$, meaning they are linear function types. The term level $\lambda$-abstractions of these types must be used exactly once.

\subsection{Term Formation} \label{teformation}
The rules for term formation are presented in Figure~\ref{term}.

\begin{figure}[h]
  \caption{Term Formation}
  \begin{mathpar}
    \inferrule
    { \pure{\Gamma_1, \Gamma_2} }
    { \Gamma_1, x \utype A, \Gamma_2 \vdash x : A }
    \rname{U-Var}

    \inferrule
    { \pure{\Gamma_1, \Gamma_2} }
    { \Gamma_1, x \ltype A, \Gamma_2 \vdash x : A }
    \rname{L-Var}

    \inferrule
    { \pure{\Gamma} \\
      \Gamma \vdash (x \stype{s} A) \rightarrow B : t_i \\
      \Gamma, x \stype{s} A \vdash n : B }
    { \Gamma \vdash \lambda x \stype{s} A . n : (x \stype{s} A) \rightarrow B }
    \rname{$\lambda$$\rightarrow$}

    \inferrule
    { \overline{\Gamma} \vdash (x \stype{s} A) \multimap B : t_i \\
      \Gamma, x \stype{s} A \vdash n : B }
    { \Gamma \vdash \lambda x \stype{s} A . n : (x \stype{s} A) \multimap B }
    \rname{$\lambda$$\multimap$}

    \inferrule
    { \Gamma_1 \vdash m : (x \utype A) \rightarrow B \\
      \pure{\Gamma_2} \\
      \Gamma_2 \vdash n : A \\
      \mrg{\Gamma_1}{\Gamma_2}{\Gamma} }
    { \Gamma \vdash m\ n : B[n/x] }
    \rname{App-U$\rightarrow$}

    \inferrule
    { \Gamma_1 \vdash m : (x \ltype A) \rightarrow B \\
      \Gamma_2 \vdash n : A \\
      \mrg{\Gamma_1}{\Gamma_2}{\Gamma} }
    { \Gamma \vdash m\ n : B[n/x] }
    \rname{App-L$\rightarrow$}

    \inferrule
    { \Gamma_1 \vdash m : (x \utype A) \multimap B \\
      \pure{\Gamma_2} \\
      \Gamma_2 \vdash n : A \\
      \mrg{\Gamma_1}{\Gamma_2}{\Gamma} }
    { \Gamma \vdash m\ n : B[n/x] }
    \rname{App-U$\multimap$}

    \inferrule
    { \Gamma_1 \vdash m : (x \ltype A) \multimap B \\
      \Gamma_2 \vdash n : A \\
      \mrg{\Gamma_1}{\Gamma_2}{\Gamma} }
    { \Gamma \vdash m\ n : B[n/x] }
    \rname{App-L$\multimap$}

    \inferrule
    { \Gamma \vdash m : A \\
      \overline{\Gamma} \vdash B : s_i \\ A \preceq B }
    { \Gamma \vdash m : B }
    \rname{Conversion}
  \end{mathpar}
  \label{term}
  \Description{}
\end{figure}

The rules \rname{U-Var}, and \rname{L-Var} are used for typing free variables. The \rname{U-Var} rule asserts that free variable $x$ occurs within the context $\Gamma_1, x \utype A, \Gamma_2$ with a non-linear type $A$. The \rname{L-Var} rule asserts that free variable $x$ occurs within the context $\Gamma_1, x \ltype A, \Gamma_2$ with linear type $A$. For both rules, the side condition $\pure{\Gamma_1, \Gamma_2}$ forbids irrelevant variables of linear types from occurring within the context. This prevents another vector of weakening variables with linear type.

In \rname{$\lambda$$\rightarrow$}, the function type being addressed has form $(x \stype{s} A) \rightarrow B$. $\lambda$-abstractions of this type can be applied an unrestricted number of times, hence cannot depend on free variables with restricted usage without possibly duplicating them. This consideration is realized by the side condition $\pure{\Gamma}$, asserting all variables in context $\Gamma$ are unrestricted. Next, the body of the abstraction $n$ is typed as $\Gamma, x \stype{s} A \vdash n : B$, where $s$ is the sort of $A$. Finally, the resulting judgment $\Gamma \vdash \lambda x.n : (x \stype{s} A) \rightarrow B$ asserts that the $\lambda$-abstraction can be used unrestrictedly.

In contrast to \rname{$\lambda$$\rightarrow$}, the \rname{$\lambda$$\multimap$} rule is used for forming $\lambda$-abstractions that must be used exactly once. Due to the fact these abstractions must be used once, they are allowed access to restricted variables within context $\Gamma$, evident in the judgment $\Gamma, x \stype{s} A \vdash n : B$, and lack of side condition $\pure{\Gamma}$. However, in judgment $\overline{\Gamma} \vdash (x \stype{s} A) \multimap B : L_k$ the context must be filtered, because types are not allowed to depend on restricted variables. The final resulting judgment $\Gamma \vdash \lambda x.n : (x \stype{s} A) \multimap B$ asserts that the $\lambda$-abstraction must be used exactly once.

For \rname{App-U$\rightarrow$}, domain $A$ is a non-linear type, as seen by its annotation in $(x \utype A) \rightarrow B$. Intuitively, this tells us that $x$ may be used an arbitrary number of times within the body of $m$. Thus, the supplied argument $n$ must not depend on restricted variables in context $\Gamma_2$. Otherwise, substitution may put multiple copies of $n$ into $m$ during $\beta$-reduction, duplicating variables that should have been restricted. This justifies the side condition of $\pure{\Gamma_2}$. The contexts $\Gamma_1$, and $\Gamma_2$ are finally merged together into $\Gamma$ by the relation $\mrg{\Gamma_1}{\Gamma_2}{\Gamma}$, contracting all unrestricted variables shared between $\Gamma_1$, and $\Gamma_2$.

Now for \rname{App-L$\rightarrow$}, domain $A$ is a linear type, as seen by its annotation in $(x \ltype A) \rightarrow B$. Intuitively, this tells us that $x$ must be used once within the body of $m$. During $\beta$-reduction, substitution will only put a single copy of $n$ into the body of $m$, so $n$ can depend on restricted variables within $\Gamma_2$ without fear of duplicating them. This justifies the lack of side condition $\pure{\Gamma_2}$. The contexts $\Gamma_1$, and $\Gamma_2$ are finally merged together into $\Gamma$ by the relation $\mrg{\Gamma_1}{\Gamma_2}{\Gamma}$, contracting all unrestricted variables shared between $\Gamma_1$, and $\Gamma_2$.

The rules \rname{App-U$\multimap$}, \rname{App-L$\multimap$} follow the same considerations as \rname{App-U$\rightarrow$}, \rname{App-L$\rightarrow$}. Additional conditions are not required for these two rules to be sound.

Finally, the \rname{Conversion} rule allows judgment $\Gamma \vdash m : A$ to convert to judgment $\Gamma \vdash m : B$ if $B$ is a valid type in context $\overline{\Gamma}$. Futhermore, $A$ must be a subtype of $B$ satisfying the relation $A \preceq B$. This rule gives rise to large eliminations (compute types from terms), as computations embedded at the type level can reduce to canonical types.

\subsection{Equality and Reduction} \label{reduction}
The operational semantics and $\beta$-equality of CLC terms are presented in Figure \ref{red}, all of which are entirely standard.

As we have discussed previously, the elimination of the explicit ! exponential allows CLC to maintain the standard operational semantics of CC$\omega$, whose $\beta$-reductions are very well behaved.


\begin{figure}[h]
  \caption{Equality and Reduction}
  \begin{mathpar}
    \inferrule
    { m_1 \step^* n \\ m_2 \step^* n }
    { m_1 \equiv m_2 : A }
    \rname{Join}

    \inferrule
    {  }
    { (\lambda x \stype{s}A.m)\ n \pstep m[n/x] }
    \rname{Step-$\beta$}

    \inferrule
    { A \step A' }
    { \lambda x \stype{s}A.m \step \lambda x \stype{s}A' .m }
    \rname{Step-$\lambda$L}

    \inferrule
    { m \step m' }
    { \lambda x \stype{s}A.m \step \lambda x \stype{s}A.m' }
    \rname{Step-$\lambda$R}

    \inferrule
    { A \pstep A' }
    { (x \stype{s} A) \rightarrow B \step (x \stype{s} A') \rightarrow B }
    \rname{Step-L$\rightarrow$}

    \inferrule
    { B \pstep B' }
    { (x \stype{s} A) \rightarrow B \step (x \stype{s} A) \rightarrow B' }
    \rname{Step-R$\rightarrow$}

    \inferrule
    { A \pstep A' }
    { (x \stype{s} A) \multimap B \step (x \stype{s} A') \multimap B }
    \rname{Step-L$\multimap$}

    \inferrule
    { B \pstep B' }
    { (x \stype{s} A) \multimap B \step (x \stype{s} A) \multimap B' }
    \rname{Step-R$\multimap$}

    \inferrule
    { m \step m' }
    { m\ n \step m'\ n }
    \rname{Step-AppL}

    \inferrule
    { n \step n' }
    { m\ n \step m\ n' }
    \rname{Step-AppR}
  \end{mathpar}
  \label{red}
  \Description{}
\end{figure}

\subsection{Meta Theory} \label{meta}
In this section, we focus our discussion on the properties of CLC. First, we show the type soundness of CLC through the \textit{subject reduction} theorem. Next, we show that CLC is an effective linear type theory through the \textit{linearity} theorem. Furthermore, we show that the \textit{promotion} and \textit{dereliction} rules can be encoded as $\eta$-expansions. Finally, we construct a reduction preserving erasure function that maps well-typed CLC terms to well-typed CC$\omega$ terms, showing that CLC is strongly normalizing.

All proofs have been formalized in Coq with help from the Autosubst \cite{autosubst} library. The Coq development is publicly available on the first author's Github repository. To the best of our knowledge, this is the first machined checked formalization of a linear dependently type theory. We give a hand written version of the proof in the appendix as well.

\subsubsection{Reduction and Confluence}

The following lemmas and proofs are entirely standard using parallel step technique \cite{takahashi}. The presence of linear types do not pose any complications as reductions are untyped.

\begin{lemma}
  Parallel reduction satisfies the diamond property. If $m \pstep_p m_1$ and $m \pstep_p m_2$ then there exists $m'$ such that $m_1 \pstep_p m'$ and $m_2 \pstep_p m'$.
\end{lemma}

\begin{lemma}
  Each $\pstep_p$ is reachable with $\step^*$. If $m \pstep_p m'$ then $m \step^* m'$.
\end{lemma}

\begin{theorem}
  The transitive reflexive closure of reduction is confluent. If $m \step^* m_1$ and $m \step^* m_2$ then there exists $m'$ such that $m_1 \pstep^* m'$ and $m_2 \pstep^* m'$.
\end{theorem}

\begin{corollary}
  The definitional equality relation $\equiv$ is an equivalence relation.
\end{corollary}

\subsubsection{Weakening} \label{weakening}
CLC restricts the weakening rule for variables of linear types. However, weakening variables of non-linear types remain admissible.

\begin{lemma}
  Weakening. If $\Gamma \vdash m : A$ is a valid judgment, then for any $x \notin \Gamma$, the judgment $\Gamma, x \utype B \vdash m : A$ is derivable.
\end{lemma}

\subsubsection{Substitution} \label{subst}
Though the substitution lemma is regarded as a boring and bureaucratic result, it is surprisingly hard to design linear typed languages where the substitution lemma is admissible. Much of the difficulty arises during the substitution of arguments containing ! exponential. Perhaps the most famous work detailing the issues of substitution is due to Wadler \cite{substitute}. He defines additional syntax and semantics for the intricate unboxing of ! terms, solving the lack of substitute in Abramsky's term calculus.

Our design of integrating ! into universe sorts removes the need for ! manipulating syntax and semantics. The following substitution lemmas are directly proved by induction on typing derivation.

\begin{lemma}
  Non-linear Substitution. For $\Gamma_1, x \utype A \vdash m : B$ and $\Gamma_2 \vdash n : A$, if $\pure{\Gamma_2}$ and $\mrg{\Gamma_1}{\Gamma_2}{\Gamma}$ are valid for some $\Gamma$, then $\Gamma \vdash m[n/x] : B[n/x]$.
\end{lemma}

\begin{lemma}
  Linear Substitution. For $\Gamma_1, x \ltype A \vdash m : B$ and $\Gamma_2 \vdash n : A$, if $\mrg{\Gamma_1}{\Gamma_2}{\Gamma}$ is valid for some $\Gamma$, then $\Gamma \vdash m[n/x] : B[n/x]$.
\end{lemma}

\subsubsection{Type Soundness}
In order to prove subject reduction, we first prove the validity theorem. The main purpose of validity is to lower types down to the term level, enabling the application of various inversion lemmas.

\begin{theorem}
  Validity. For any context $\Gamma$, term $m$, type $A$, and sort $s$, if $\Gamma \vdash m : A$ is a valid judgment, then there exist some sort $s$ and level $i$ such that $\overline{\Gamma} \vdash A : s_i$.
  \label{validity}
\end{theorem}

With weakening, substitution, propagation and various inversion lemmas proven, subject reduction can now be proved by induction on typing derivation.

\begin{theorem}
  Subject Reduction. For $\Gamma \vdash m : A$, if $m \pstep n$ then $\Gamma \vdash n : A$.
\end{theorem}

\subsubsection{Linearity}
At this point, we have proven that CLC is type sound. However, we still need to prove that the removal of weakening and contraction for restricted variables yields tangible impact on the structure of terms. For this purpose, we define a binding aware recursive function $occurs(x, m)$ that counts the number of times variable $x$ appears within term $m$. The linearity theorem asserts that restricted variables are used exactly once within a term, subsuming safe resource usage.

\begin{theorem}
  Linearity. If $\Gamma \vdash m : A$ is a valid judgment, for any $(x \ltype B) \in \Gamma$, there is $occurs(x, m) = 1$.
\end{theorem}

\subsubsection{Promotion and Dereliction}
Linear types of CLC are not obtained through packing and unpacking !, so there are no explicit rules for \textit{promotion} and \textit{dereliction} of Linear Logic. Arbitrary computations existing at the type level also muddle the association between which types can be promoted or derelicted to which. Nevertheless, \textit{promotion} and \textit{dereliction} for canonical types are derivable as theorems through $\eta$-expansion.

\begin{theorem}
  Promotion. If $\Gamma \vdash m : (x \stype{s} A) \multimap B$ and $\pure{\Gamma}$ are valid judgments, then there exists $n$ such that $\Gamma \vdash n : (x \stype{s} A) \rightarrow B$ is derivable.
\end{theorem}

\begin{theorem}
  Dereliction. If $\Gamma \vdash m : (x \stype{s} A) \rightarrow B$ is a valid judgment, then there exists $n$ such that $\Gamma \vdash n : (x \stype{s} A) \multimap B$ is derivable.
\end{theorem}

\subsubsection{Strong Normalization}
The strong normalization theorem of CLC is proven by construction of a typing and reduction preserving erasure function to CC$\omega$. We assume familiarity with CC$\omega$ syntax here, and define the erasure function as follows.

\begin{definition}
  \begin{align*}
    \erase{x}                             & = x                                     \\
    \erase{U_i}                           & = Type_i                                \\
    \erase{L_i}                           & = Type_i                                \\
    \erase{(x \stype{s} A) \rightarrow B} & = (x : \erase{A}) \rightarrow \erase{B} \\
    \erase{(x \stype{s} A) \multimap B}   & = (x : \erase{A}) \rightarrow \erase{B} \\
    \erase{\lambda x\stype{s}A.n}         & = \lambda x : \erase{A}.\erase{n}       \\
    \erase{m\ n}                          & = \erase{m}\ \erase{n}
  \end{align*}
\end{definition}

With slight overloading of notation, we define erasure for CLC contexts recursively.

\begin{definition}
  \begin{align*}
    \erase{\epsilon}              & = \epsilon                      \\
    \erase{\Gamma, x \stype{s} A} & = \erase{\Gamma}, x : \erase{A}
  \end{align*}
\end{definition}

We prove the following lemma to commute erasure and substitution whenever needed.

\begin{lemma}
  Erasure commutes with substitution. For any CLC terms $m$, $n$ and some variable $x$, $\erase{m[n/x]} = \erase{m}[\erase{n}/x]$.
\end{lemma}

For the following lemma, we refer to the reductions in CLC as $\step_{\scriptscriptstyle \text{CLC}}$, and the reductions in CC$\omega$ as $\step_{\scriptscriptstyle \text{CC$\omega$}}$.

\begin{theorem} \label{preserve}
  Reduction Erasure. For any CLC terms $m$ and $n$, if there is $m \step_{\scriptscriptstyle \text{CLC}} n$, then there is $\erase{m} \step_{\scriptscriptstyle \text{CC$\omega$}} \erase{n}$.
\end{theorem}

\begin{theorem} \label{embedding}
  Embedding. If $\Gamma \vdash m \stype{s} A$ is a valid judgment in CLC, then $\erase{\Gamma} \vdash \erase{m} : \erase{A}$ is a valid judgment in CC$\omega$.
\end{theorem}

If there exists some well typed CLC term with an infinite sequence of reductions, erasure will embed this term into a well typed CC$\omega$ term along with its infinite sequence of reductions by virtue of theorems \ref{preserve}, and \ref{embedding}. This is contradictory to the strong normalization property of CC$\omega$ proven through the Girard-Tait method \cite{ecc}, so this hypothetical term does not exist in CLC.

\begin{theorem}
  Well-typed CLC terms are strongly normalizing.
\end{theorem}

\subsubsection{Backwards Compatibility}
To show that CLC is backwards compatible with the predicative CC$\omega$, we construct a function that annotates CC$\omega$ terms with sort $U$, lifting them into the non-linear fragment of CLC.

\begin{definition}
  \begin{align*}
    \lift{x}                     & = x                                        \\
    \lift{Type_i}                & = U_i                                      \\
    \lift{(x : A) \rightarrow B} & = (x \utype \lift{A}) \rightarrow \lift{B} \\
    \lift{\lambda x : A.n}       & = \lambda x\utype\lift{A}.\lift{n}         \\
    \lift{m\ n}                  & = \lift{m}\ \lift{n}
  \end{align*}
\end{definition}

With slight overloading of notation, we define lifting for CC$\omega$ recursively.
\begin{definition}
  \begin{align*}
    \lift{\epsilon}      & = \epsilon                         \\
    \lift{\Gamma, x : A} & = \lift{\Gamma}, x \utype \lift{A}
  \end{align*}
\end{definition}

The following lemmas and theorems are all proved following a similar procedure to proving the soundness of erasure.
\begin{lemma}
  Lift commutes with substitution. For any CC$\omega$ terms $m$, $n$ and some variable $x$, $\lift{m[n/x]} = \lift{m}[\lift{n}/x]$.
\end{lemma}

\begin{theorem}
  Lift preserves reduction. For any CC$\omega$ term $m$ and $n$, if there is $m \step_{\scriptscriptstyle \text{CC}\omega} n$, then there is $\lift{m} \step_{\scriptscriptstyle \text{CLC}} \lift{n}$.
\end{theorem}

\begin{theorem} \label{lift}
  Lifting. If $\Gamma \vdash m : A$ is a valid judgment in CC$\omega$, then $\lift{\Gamma} \vdash \lift{m} : \lift{A}$ is a valid judgment in CLC.
\end{theorem}

The lifting theorem shows that all valid CC$\omega$ terms are valid CLC terms if the domain of function types are annotated with sort $U$. In practice, these annotations can often be omitted if algorithmic type checking is able to infer them from context.

\section{The Extension to CILC}
Although the core CLC language is extremely expressive, formally shown by Theorem \ref{lift} to be at least as expressive as the predicative CC$\omega$, the complexity of Church-style encodings to do so will quickly become unmanageable. To address this, we extend CLC with rules for defining inductive linear types in the style of CIC.

We have formalized CILC in our Coq development and proven its soundness.

\subsection{Syntax of CILC}
\begin{figure}[h]
  \caption{Syntax}
  \begin{grammar}
    <$C,I,P,Q,f$> ::= ... \hspace*{10em} expressions
    \indalt $\ind{s}(X : A)\{C_1 \sep ... \sep C_k\}$
    \indalt $\constr(k, I)$
    \indalt $\case(m, Q)\{f_1 \sep ... \sep f_k \}$
    \indalt $\dcase(m, Q)\{f_1 \sep ... \sep f_k \}$
    \indalt $\fix f : A := m$
  \end{grammar}
  \Description{}
  \label{syntax2}
\end{figure}

As shown in Figure \ref{syntax2}, CILC extends CIC with for the introduction and elimination of inductive types. Ind is used during the formation of new inductive types, and Constr is used for introducing their constructors. Case and DCase are non-dependent and dependent case eliminators respectively. CIC-style primitive recursion has been factored out of case eliminators in favor of a Coq-style Fix construct. The flexibility of an independent Fix construct allows for a cleaner formalization of inductive linear type elimination.

\subsection{Arity}
For type $A$ and sort $s$, the judgment $arity(A, s)$ is inductively defined over the structure of $A$ as depicted in Figure \ref{arity}.

\begin{figure}[h]
  \caption{Arity}
  \begin{mathpar}
    \inferrule
    { }
    { arity(s_i,s) }
    \rname{A-Sort}

    \inferrule
    { arity(A, s) }
    { arity((x \utype M) \rightarrow A, s)}
    \rname{A$\rightarrow$}
  \end{mathpar}
  \Description{}
  \label{arity}
\end{figure}

\begin{definition}
  For some arity type $A$ and sort $s'$, the term $A\lrangle{s'}$ is inductively defined over the structure of $A$ as follows.
  \begin{align*}
    ((x \utype M) \rightarrow A)\lrangle{s'} & = (x \utype M) \rightarrow A\lrangle{s'} \\
    (s)\lrangle{s'}                          & = s'
  \end{align*}
\end{definition}

\begin{definition}
  For some arity type $A$, term $I$ and sort $s'$, the term $A\{I, s'\}$ is inductively defined over the structure of $A$ as follows.
  \begin{align*}
    ((x \utype M) \rightarrow A)\lrangle{I, s'} & = (x \utype M) \rightarrow A\lrangle{(I\ x), s'} \\
    (s)\lrangle{I, s'}                          & = (\_ \utype I) \rightarrow s'
  \end{align*}
\end{definition}

\subsection{Strict Positivity}
For type $P$ and variable $X$, the judgment $postive(P, X)$ is inductively defined over the structure of $P$ as depicted in Figure \ref{positive}.

We say that $X$ occurs strictly positive in $P$ if the judgment $postive(P,X)$ is valid.

\begin{figure}[h]
  \caption{Strict Positivity}
  \begin{mathpar}
    \inferrule
    { (\forall i = 1...k) \\ X \notin m_i }
    { positive((X\ m_1...m_k), X) }
    \rname{Pos-X}

    \inferrule
    { X \notin M \\ positive(P, X)}
    { positive((x \stype{s} M) \rightarrow P, X) }
    \rname{Pos$\rightarrow$}

    \inferrule
    { X \notin M \\ positive(P, X) }
    { positive((x \stype{s} M) \multimap P, X) }
    \rname{Pos$\multimap$}
  \end{mathpar}
  \Description{}
  \label{positive}
\end{figure}

\subsection{Non-Linear Constructor}
For type $C$ and variable $X$, the judgment $\ucons(C, X)$ is inductively defined over the structure of $C$ as shown in Figure \ref{nlconstr}.

Non-linear constructors are used for constructing non-linear objects that may be freely used. These constructors may not applied to restricted terms as this may cause duplication.

\begin{figure}[h]
  \caption{Non-linear Constructor}
  \begin{mathpar}
    \inferrule
    { (\forall i = 1...k) \\ X \notin m_i }
    { \ucons((X\ m_i...m_k), X) }
    \rname{U-Constr-X}

    \inferrule
    { positive(P, X) \\ \ucons(C, X) }
    { \ucons((\_ \utype P) \rightarrow C, X)}
    \rname{U-Constr-pos}

    \inferrule
    { X \notin M \\ \ucons(C, X) }
    { \ucons((x \utype M) \rightarrow C, X)}
    \rname{U-Constr$\rightarrow$}
  \end{mathpar}
  \Description{}
  \label{nlconstr}
\end{figure}

\subsection{Linear Constructor}
For type $C$ and variable $X$, the judgment $\lcons(C, X)$ is inductively defined over the structure of $C$ as shown in Figure \ref{lconstr}.

Linear constructors are used for constructing linear objects that must be used once. These constructors may be applied to linear terms as restricted usage prevent duplication and enforce single usage.

An unapplied linear constructor is a non-linear entity as they can be created freely ``out of thin air". However, once a linear constructor has been partially applied to a linear term, its linearity is ``activated'' in rules \rname{L-Constr-Pos-2} and \rname{L-Constr-2$\rightarrow$}, essentially forcing the rest of its arguments to be fully applied. This is to prevent partially applied constructors from duplicating linear variables.

\begin{figure}[h]
  \caption{Linear Constructor}
  \begin{mathpar}
    \inferrule
    { (\forall i = 1...k) \\ X \notin m_i }
    { \lcons((X\ m_i...m_k), X) }
    \rname{L-Constr-X}

    \inferrule
    { positive(P, X) \\ \lcons(C, X) }
    { \lcons((\_ \utype P) \rightarrow C, X)}
    \rname{L-Constr-Pos-1}

    \inferrule
    { X \notin M \\ \lcons(C, X) }
    { \lcons((x \utype M) \rightarrow C, X)}
    \rname{L-Constr-1$\rightarrow$}

    \inferrule
    { positive(P, X) \\ activation(C, X) }
    { \lcons((\_ \ltype P) \rightarrow C, X)}
    \rname{L-Constr-Pos-2}

    \inferrule
    { X \notin M \\ activation(C, X) }
    { \lcons((x \ltype M) \rightarrow C, X)}
    \rname{L-Constr-2$\rightarrow$}
  \end{mathpar}
  \Description{}
  \label{lconstr}
\end{figure}

For type $C$ and variable $X$, the judgment $activation(C, X)$ is inductively defined over the structure of $C$ as shown in Figure \ref{activation}. Intuitively, $activation(C,X)$ ensures the linearity of $C$.

\begin{figure}[h]
  \caption{Activation}
  \begin{mathpar}
    \inferrule
    { (\forall i = 1...k) \\ X \notin m_i }
    { activation((X\ m_i...m_k), X) }
    \rname{Act-X}

    \inferrule
    { positive(P, X) \\ activation(C, X) }
    { activation((\_ \stype P) \multimap C, X)}
    \rname{Act-Pos}

    \inferrule
    { X \notin M \\ activation(C, X) }
    { activation((x \stype M) \multimap C, X)}
    \rname{Act$\multimap$}
  \end{mathpar}
  \Description{}
  \label{activation}
\end{figure}

\subsection{Introduction Rules}
The formation of new inductive types and their constructors are presented in Figure \ref{ind}.

\begin{figure}[h]
  \caption{Inductive Type Formation}
  \begin{mathpar}
    \inferrule
    { (\forall j = 1...n) \\ arity(A, s) \\ \scons(C_j, X) \\\\
      \pure{\Gamma} \\
      \Gamma \vdash A : U_i \\
      \Gamma, X \utype A \vdash C_j : t_i }
    { \Gamma \vdash \ind{s}(X : A)\{C_1|...|C_k\} : A }
    \rname{Ind}

    \inferrule
    { 1 \leq i \leq k \\\\
    \pure{\Gamma} \\
    I := \ind{s}(X : A)\{C_1|...|C_k\} \\
    \Gamma \vdash I :A }
    { \Gamma \vdash \constr(i, I) : C_i[I/X] }
    \rname{Constr}
  \end{mathpar}
  \Description{}
  \label{ind}
\end{figure}

\subsection{Non-Dependent Case}

\begin{definition}
  For constructor type $C$ and variables $X$ and $Q$, the term $C\{X, Q\}$ is inductively defined over the structure of $C$ as follows.
  \begin{align*}
    ((\_ \stype{s} P) \rightarrow C)\{X, Q\} & = (\_ \stype{s} P) \multimap C\{X, Q\} \\
    ((\_ \stype{s} P) \multimap C)\{X, Q\}   & = (\_ \stype{s} P) \multimap C\{X, Q\} \\
    ((x \stype{s} M) \rightarrow C)\{X, Q\}  & = (x \stype{s} P) \multimap C\{X, Q\}  \\
    ((x \stype{s} M) \multimap C)\{X, Q\}    & = (x \stype{s} P) \multimap C\{X, Q\}  \\
    (X\ m_1...m_k)\{X, Q\}                   & = (Q\ m_1...m_k)
  \end{align*}
  We define the notation $C[I, P] := C\{X, Q\}[I/X, P/Q]$.
\end{definition}

The rule for non-dependent case elimination is presented in Figure \ref{case}. This elimination rule is non-dependent because the motive $Q$ may not depend on the discriminated term $m$ itself.

\begin{figure}[h]
  \caption{Non-dependent Case Elimination}
  \begin{mathpar}
    \inferrule
    { (\forall i = 1...k_1) \\
    \mrg{\Gamma_1}{\Gamma_2}{\Gamma} \\
    arity(A, s) \\\\
    I := \ind{s}(X : A)\{C_1|...|C_{k_1}\} \\
    \overline{\Gamma_2} \vdash Q : A\lrangle{s'} \\\\
    \Gamma_1 \vdash m : (I\ a_1...a_{k_2}) \\
    \Gamma_2 \vdash f_i : C_i[I, Q] }
    { \Gamma \vdash \case(m, Q)\{f_1|...|f_{k_1}\} : (Q\ a_1...a_{k_2}) }
    \rname{Case}
  \end{mathpar}
  \Description{}
  \label{case}
\end{figure}

\subsection{Dependent Case}

\begin{definition}
  For some non-linear constructor type $C$ and type variables $X$, $Q$ and $c$, the term $C\{X, Q, c\}$ is inductively defined over the structure of $C$ as follows.
  \begin{align*}
    ((\_ \utype P) \rightarrow C)\{X, Q, c\} & = (p \utype P) \multimap C\{X, Q, (c\ p)\} \\
    ((x \utype M) \rightarrow C)\{X, Q, c\}  & = (x \utype M) \multimap C\{X, Q, (c\ x)\} \\
    (X\ m_1...m_n)\{X, Q, c\}                & = (Q\ m_1...m_n\ c)
  \end{align*}
  We define the notation $C[I, P, t] := C\{X, Q, c\}[I/X, P/Q, t/c]$.
\end{definition}

Dependent elimination of non-linear inductive objects is presented in Figure \ref{dcase}. This elimination rule is dependent because the motive $Q$ may depend on the discriminated term $m$. For this very reason, $m$ is not allowed to have linear type as doing so may transport restricted variable into types.

\begin{figure}[h]
  \caption{Dependent Case Elimination}
  \begin{mathpar}
    \inferrule
    { \pure{\Gamma_1} \\
    (\forall i = 1...n) \\
    \mrg{\Gamma_1}{\Gamma_2}{\Gamma} \\
    arity(A, U_i) \\\\
    I := \ind{U}(X : A)\{C_1|...|C_n\} \\
    \overline{\Gamma_2} \vdash Q : A\lrangle{I, s'} \\\\
    \Gamma_1 \vdash m : (I\ a_1...a_n) \\
    \Gamma_2 \vdash f_i : C_i[I, Q, Constr(i, I)] }
    { \Gamma \vdash \dcase(m, Q)\{f_1|...|f_n\} : (Q\ a_1...a_n\ m) }
    \rname{DCase}
  \end{mathpar}
  \Description{}
  \label{dcase}
\end{figure}

\subsection{Fixpoint}
The typing rule of fix-points presented in Figure \ref{fix} is mostly standard. A syntactic guard condition is utilized to conservatively ensure termination. The fix-point term must have non-linear type in a pure context, otherwise recursive calls may cause duplicated usage of restricted variables.

\begin{figure}[h]
  \caption{Fixpoint}
  \begin{mathpar}
    \inferrule
    { \pure{\Gamma} \\
      \Gamma \vdash A : U_i \\
      \Gamma, f \utype A \vdash m : A \\
      \text{guard condition} }
    { \Gamma \vdash \fix f : A := m : A }
    \rname{fix}
  \end{mathpar}
  \Description{}
  \label{fix}
\end{figure}

\subsection{Extended Reduction}
The reduction semantics for the extended language is standard \cite{cic}, we only present the most interesting $\iota$-reduction cases.

\begin{figure}[h]
  \caption{$\iota$-Reductions}
  \begin{mathpar}
    \inferrule
    { }
    { \case((\constr(i, I)\ a_1...a_n),Q)\{f_1|...f_k\} \\\\
      \leadsto (f_i\ a_1...a_k) }
    \rname{Step-Case-$\iota$}

    \inferrule
    { }
    { \dcase((\constr(i, I)\ a_1...a_k),Q)\{f_1|...f_k\} \\\\
      \leadsto (f_i\ a_1...a_k) }
    \rname{Step-DCase-$\iota$}

    \inferrule
    { }
    { \fix f : A := m \leadsto m[(\fix f : A := m)/f] }
    \rname{Step-Fix-$\iota$}
  \end{mathpar}
  \Description{}
  \label{stepfix}
\end{figure}

\subsection{Non-Linear Connectives}
Non-linear inductive types have been extensively studied since their inception. We will omit discussion of non-linear inductive types here as all inductive definitions in the predicative fragment of CIC are available in CILC.

\subsection{Linear Connectives}
In Figure \ref{lconnectives}, we demonstrate how the $\otimes$ and $\oplus$ connectives of linear logic could be defined in CILC as inductive linear types. For any term with inductive linear type, linearity will ultimately force the term to be eliminated by a Case expression.

\begin{figure}[h]
  \caption{Inductively Defined Linear Connectives}
  \begin{align*}
    \_\otimes\_ :=\
     & \ind{L}(X : (A \utype L_i) \rightarrow (B \utype L_i) \rightarrow L_i)\{      \\
     & \sep\ (A \utype L_i) \rightarrow (B \utype L_i)                               \\
     & \qquad \rightarrow (\_ \ltype A)  \multimap (\_ \ltype B) \multimap (X\ A\ B) \\
     & \}
    \\
    \lrangle{\_,\_} := \
     & \constr(1, \otimes)
    \\\\
    \_\oplus\_ := \
     & \ind{L}(X : (A \utype L_i) \rightarrow (B \utype L_i) \rightarrow L_i)\{      \\
     & \sep \ (A \utype L_i) \rightarrow (B \utype L_i)
    \rightarrow (\_ \ltype A)  \multimap (X\ A\ B)                                   \\
     & \sep \ (A \utype L_i) \rightarrow (B \utype L_i)
    \rightarrow (\_ \ltype B)  \multimap (X\ A\ B)                                   \\
     & \}
    \\
    \inl(\_) := \
     & \constr(1, \oplus)
    \\
    \inr(\_) := \
     & \constr(2, \oplus)
  \end{align*}
  \Description{}
  \label{lconnectives}
\end{figure}

\subsection{Mixed Connectives}
In Figure \ref{mconnectives}, we demonstrate how the $F$ connective of LNL$_D$ \cite{neel15} can be expressed as an inductive linear type. The interesting property of $F$ is that the type of its linear component is dependent on the non-linear component.

\begin{figure}[h]
  \caption{Linear Non-Linear Connective}
  \begin{align*}
    F := \
     & \ind{L}(X : (A \utype U_i) \rightarrow
    (B \utype (\_ \utype A) \rightarrow L_i) \rightarrow L_i)\{ \\
     & \sep \ (A \utype U_i) \rightarrow
    (B \utype (\_ \utype A) \rightarrow L_i)                    \\
     & \qquad \rightarrow (m \utype A) \rightarrow
    (\_ \ltype B\ m) \multimap (X\ A\ B)                        \\
     & \}
    \\
    [\_,\_] :=\
     & \constr(1, F)
  \end{align*}
  \Description{}
  \label{mconnectives}
\end{figure}

\section{Examples}
We shall write examples in the concrete syntax of our implementation. We deliberately chose to model our concrete syntax after Coq such that it would be easier for readers already familiar with Coq to adapt to our syntax.

\subsection{Stateful Reasoning}
We first postulate five state axioms. These axioms could be viewed as an interface to a trusted memory allocator.
\begin{definition}
  State axioms.
  \begin{itemize}
    \item $l \mapsto_A m : \mathbb{N} \rightarrow (A : U) \rightarrow A \rightarrow L$ \\
          The $\mapsto$ is a linear type constructor. It fulfills a role similar to L3's \cite{l3} memory access capability, or Separation Logic's \cite{reynolds02} points-to assertion. Intuitively, it is a proof that a term $m$ of non-linear type $A$ is currently stored at address $l$.
    \item $\textbf{new} : (A : U) \rightarrow (m : A) \rightarrow F l:\mathbb{N}.(l \mapsto_A m)$ \\
          $\new$ is a function that allocates a new memory cell for term $m$ of non-linear type $A$. It results in an address $l$ paired with an assertion $(l \mapsto_A m)$.
    \item $\textbf{free} : (A : U) \rightarrow F l:\mathbb{N}.F m:A.(l \mapsto_A m) \rightarrow \unit$ \\
          $\free$ is a function that de-allocates the memory cell at location $l$ which is currently in use. This can be seen in its input $F l:\mathbb{N}.F m:A.(l \mapsto_A m)$, where the assertion component $(l \mapsto_A m)$ is a proof that location $l$ is currently allocated.
    \item $\textbf{get} : (A : U) \rightarrow (l : \mathbb{N}) \rightarrow (m : A) \rightarrow (l \mapsto_A m) \rightarrow F \_: (\Sigma x. x =_A m). (l \mapsto_A m)$ \\
          $\get$ is a function that retrieves the term stored at address $l$ when given a proof $(l \mapsto_A m)$ that there is indeed a term $m$ currently stored there. The type of $\get$'s output is the rather complicated $F \_: (\Sigma x. x =_A m). (l \mapsto_A m)$. This not only returns the term stored at $l$, but also a proof that the returned term is exactly the same as what $(l \mapsto_A m)$ claimed to have stored. The proof $(l \mapsto_A m)$ is returned unchanged as well.
    \item $\textbf{set} : (A\ B : U) \rightarrow (l : \mathbb{N}) \rightarrow (m : A) \rightarrow (l \mapsto_A m) \multimap (n : B) \multimap (l \mapsto_B n)$ \\
          $\set$ has the most complicated type of all the axioms presented here. Intuitively, if address $l$ is allocated and storing some term $m$ of type $A$, $\set$ may put term $n$ of type $B$ into the the store at address $l$, overwriting the original term. This is the so called "strong update" operation. The consumption of $(l \mapsto_A m)$ and return of $(l \mapsto_B n)$ accurately characterize this update process.
  \end{itemize}
\end{definition}
Due to the fact that $(l \mapsto_A m)$ is an abstract linear type, proofs of this type can not be duplicated and may only be consumed by other state axioms. This is the same non-sharing principle that empowers Separation Logic. Unlike Separation Logic, $(l \mapsto_A m)$ itself is a first-class object in CILC, meaning that it can be the result of computation (large eliminations), input to computations, stored in data structures, etc.

The combination of linear types and dependent types within the state axioms enable the internalization of the so called "effect laws" for state, where these laws are now provable theorems within the CILC framework.

\begin{enumerate}
  \item Performing $\get$ twice on the same address will produce two terms that are provably equal.
  \item After using $\set$ to update the store at an address $l$ to term $m$, performing $\get$ again on the same address will retrieve $m$.
\end{enumerate}

Figure \ref{law1} is an excerpt taken from our implementation. It demonstrates the internalization of Law 1. The usage of implicit parameters allows one to omit writing much of the redundant arguments that could be inferred from context. Here, $\new$ allocates a store for natural number 1, creating a proof $c$ of the form $(l \mapsto_\mathbb{N} 1)$. When $\get$ is applied to $c$ twice, the equality proofs \textit{pf1} $: (1 =_\mathbb{N} x)$ and \textit{pf2} $: (1 =_\mathbb{N} y)$ generated by $\get$ allow one to create proof \textit{pf} $: (x =_\mathbb{N} y)$ that the retrieved values $x$ and $y$ are equal. Finally, due to $c$ having linear type, it is required for it to be freed at the very end.

\begin{figure}[h]
  \caption{Law 1}
  \begin{minted}{coq}
  let [ _, c ] := new _ 1 in
  let [ xeq, c ] := get _ _ _ c in
  let [ yeq, c ] := get _ _ _ c in
  let ( x, pf1 ) := xeq in
  let ( y, pf2 ) := yeq in
  let pf1 := eq_sym _ _ _ pf1 in
  let pf : eq _ x y := 
    eq_trans _ _ _ _ pf1 pf2 
  in free _ _ _ c.
  \end{minted}
  \label{law1}
  \Description{}
\end{figure}

Figure \ref{law2} demonstrates the internalization of Law 2. Again, implicit parameters fill in the annoying bureaucratic arguments whenever they can be inferred. Here, a store is allocated by $\new$ for natural number 1, creating a proof $c$ of the form $(l \mapsto_\mathbb{N} 1)$. After using $\set$ to update the store at $l$ to 2 and then performing $\get$ on $l$, one can create proof \textit{pf} $: (2 =_\mathbb{N} z)$ that the value $z$ retrieved by $\get$ is exactly equal to 2.

\begin{figure}[h]
  \caption{Law 2}
  \begin{minted}{coq}
  let [ _, c ] := new _ 1 in
  let c := set _ _ _ _ c 2 in
  let [ zeq, c ] := get _ _ _ c in
  let ( z, pf1 ) := zeq in
  let pf : eq _ 2 z := pf1 in
  free _ _ _ c.
  \end{minted}
  \label{law2}
  \Description{}
\end{figure}

\subsection{Protocol Enforcement}
The inclusion of dependent types and linear types in CILC makes it extremely expressive when encoding communication protocols. By indexing each channel with a protocol, CILC can enforce communication on a channel to strictly adhere to its specified protocol.

In Figure \ref{session}, we declare an inductive \texttt{session} type for specifying protocols. Intuitively, the \texttt{session}  forms a stack of possible operations executed in a protocol. The $\SEND$ constructor takes in a non-linear type $A$ and a session $ss$ as its argument, indicating that after sending a message of type $A$, the protocol progresses to stage $ss$. Likewise, the $\RECV$ constructor takes in a non-linear type $A$ and a session $ss$ as its argument, indicating that after receiving a message of type $A$, the protocol progresses to stage $ss$. Finally, the $\END$ constructor indicates that the protocol has finished and communication is over.

\begin{figure}[h]
  \caption{Inductive Session Type}
  \begin{minted}{coq}
  Inductive session : U :=
  | SEND : U -> session -> session
  | RECV : U -> session -> session
  | END : session.
  \end{minted}
  \label{session}
  \Description{}
\end{figure}

In Definition \ref{comm}, we postulate five axioms for communication using the $\session$ type. These axioms could be viewed as an interface to trusted communication libraries.

\begin{definition}
  Communication axioms.
  \begin{itemize}
    \item $\textbf{channel} : \session \rightarrow L$ \\
          The $\channel$ axiom is a type constructor. It is used to form the types of communication channels that obey the protocol specified by its session.
    \item $\textbf{open} : (ss : \session) \rightarrow \channel\ ss$ \\
          When given a protocol, $\open$ creates a channel indexed by this protocol.
    \item $\textbf{close} : \channel\ \END \rightarrow \unit$ \\
          Once all communications specified on a given channel has finished, $\close$ will close and reclaim the channel.
    \item $\textbf{send} : (A : U) \rightarrow A \rightarrow (ss : \session) \rightarrow$ \\
          \phantom{send : (A :)}
          $\channel\ (\SEND\ A\ ss) \rightarrow \channel\ ss$ \\
          After a channel has reached the point in its protocol where a message of type $A$ should be sent, $\send$ may send a term of type $A$ on this channel, causing the channel to progress to the next stage of its protocol.
    \item $\textbf{recv} : (A : U) \rightarrow (ss : \session) \rightarrow $ \\
          \phantom{recv : (A :)}
          $\channel\ (\RECV\ A\ ss) \rightarrow F \_:A.\channel\ ss$
          After a channel has reached the point in its protocol where a message of type $A$ should be received, $\recv$ may receive such a term of type $A$ from the channel, causing the channel to progress to the next stage of its protocol.
  \end{itemize}
  \label{comm}
\end{definition}

The important detail to notice here is that each channel created by $\open$ is of linear type $(\channel\ ss)$ where $ss$ is some arbitrary protocol, hence must be used exactly once. However, the $\send$ and $\recv$ operations return back channels after consuming them, progressing their protocols one stage forward. Coupled with the fact that the $\close$ operation is only able to close channels whose protocols have ended, this forces channels to communicate strictly by their specified protocols.

Dependent types also allows for more precise specifications of protocols themselves. A simple example would be the specification of message sizes. Figure \ref{communication} shows such an example. The inductive type \texttt{ilist} is a length indexed list. Using \texttt{ilist}, we construct a protocol \texttt{ss} that first sends two natural numbers, then receives eight natural numbers and finally sends back a boolean acknowledgement.

\begin{figure}[h]
  \caption{Sized Communication}
  \begin{minted}{coq}
  Inductive ilist (A : U) : Nat -> U :=
  | nil  : ilist A 0
  | cons : A -> (n : Nat) -> 
             ilist A n -> ilist A (S n).

  let ss := 
    SEND (ilist Nat 2) 
      (RECV (ilist Nat 8) (SEND Bool END)) 
  in
  let send_msg : ilist Nat 2 := 
    (cons 1 _ (cons 2 _ nil)) 
  in
  let ch := open ss in
  let ch := send _ send_msg _ ch in
  let [ recv_msg, ch ] := recv _ _ ch in
  let ch := send _ true _ ch in
  close ch.            
  \end{minted}
  \label{communication}
  \Description{}
\end{figure}

\subsection{Mutable Data}
Figure \ref{list} presents a CPS append function for linear lists. In the \texttt{cons} case, the constructor is opened, exposing its data element \texttt{h} and tail list \texttt{t}. Instead of de-allocating the memory of \texttt{cons}, it is bound to the variable \texttt{_cons_} of linear type $A \multimap list\ A \multimap list\ A$. Each \texttt{_cons_} is applied exactly once to reconstruct a list using the memory of the original, thus mutating the original list in-place. Also notice that the continuation $k$ is of linear type $(list\ A \multimap list\ A)$, indicating that $k$ must be applied exactly once. Using linear types for continuations is surprisingly accurate as control-flow itself is a linear concept.

\begin{figure}[h]
  \caption{Linear Lists}
  \begin{minted}{coq}
  Inductive list (A : U) : L :=
  | nil : list A
  | cons : A -> list A -> list A.
  
  Fixpoint append (A : U) : list A -o 
    list A -o (list A -o list A) -o list A 
  :=
    fun ls1 ls2 k =>
      match ls1 with
      | nil => k ls2
      | (cons h t) as _cons_ =>
        append _ t ls2 
          (fun res => k (_cons_ h res))
      end.
  \end{minted}
  \label{list}
  \Description{}
\end{figure}

\section{Related Work}
Linear types are a class of type systems inspired by Girard' substructural Linear Logic \cite{girard}. Girard notices that the weakening and contraction rules of Classical Logic when restricted carefully, give rise to a new logical foundation for reasoning about resource. Wadler \cite{wadler1990,wadler1991} then applies an analogous restriction to variable usage in simple type theory, leading to the development of linear type theory where terms respect resources. A term calculus for linear type theory was later realized by Abramsky \cite{abramsky1993}. Benton \cite{benton1994} investigates the ramifications of the ! exponential in linear term calculi, decomposing it to adjoint connectives $F$ and $G$ that map between linear and non-linear judgments. Programming languages \cite{l3,ats,linear-haskell} featuring linear types have also been implemented, allowing programmers to write resource safe software in practical applications.

Over the years, work has been done to enrich linear type theories with dependent types. Cervesato and Pfenning extend the Edinburgh Logical Framework with linear types \cite{lf,llf}, being the first to demonstrate that dependent types and linear types can coexist within a type theory. V\'{a}k\'{a}r \cite{vakar14} presents a linear dependent type theory, with syntax and semantics drawing inspiration from DILL \cite{dill}.  Krishnaswami et al. present a dependent linear type theory \cite{neel15} based on Benton's earlier work on mixed linear and non-linear calculus, demonstrating the ability to internalize imperative programming in the style of Hoare Type Theory \cite{htt}. Luo et al. \cite{luo} introduce the property of essential linearity and a mixed linear/non-linear context, describing the first type theory that allows types to depend on linear terms. Based on initial ideas of McBride \cite{nothing}, Atkey's Quantitative Type Theory (QTT) \cite{qtt} uses semi-ring annotations to track variable occurrence, simulating irrelevance, linear and affine types within a unified framework. The Idris 2 programming language \cite{idris2} implements QTT as its core type system.

\section{Future Work}
On the theoretical side, we intend to investigate the semantic models of CLC. In early versions of CLC, an impredicative $Prop$ universe sat at the bottom of the universe hierarchy. This has since been removed due to the additional difficulty of working with $Prop$ in the soundness proof of CILC. We believe that these difficulties are not intrinsic and $Prop$ could be added back with careful proof work.

On the application side, we intend to bolster our implementation with more features such as compilation to executable binary, making it a full-featured theorem prover/programming language. We aim to fully verify the typechecking algorithm employed by the current implementation. The unification algorithm used for implicit argument resolution is extremely ad-hoc, as the problem of unification for linear types has been not thoroughly studied, we hope to improve this situation as well.


\bibliographystyle{acm}
\bibliography{../ref}
\end{document}
