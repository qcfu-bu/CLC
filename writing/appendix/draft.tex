\documentclass{article}

\usepackage[utf8]{inputenc}
\usepackage[margin=1in]{geometry}

% packages
\usepackage{graphicx}
\usepackage{amsthm}
\usepackage{amsmath}
\usepackage{amssymb}
\usepackage{mdwtab}
\usepackage{syntax}
\usepackage{stmaryrd}
\usepackage{mathtools}
\usepackage{mathpartir}
\usepackage{listings}
\usepackage{float}
\usepackage{tikz-cd}
\usepackage{enumitem}
\usepackage{minted}

\newtheorem{theorem}{Theorem}[section]
\newtheorem{corollary}{Corollary}[theorem]
\newtheorem{lemma}[theorem]{Lemma}
\theoremstyle{definition}
\newtheorem{definition}{Definition}[section]

\setlist[itemize]{leftmargin=1.6em}
\renewcommand{\syntleft}{}
\renewcommand{\syntright}{}
\setlength{\grammarparsep}{15pt}
\setlength{\grammarindent}{10em}
\newcommand{\indalt}[1][2]{\\\hspace*{-1.2em}\textbar\quad}
\newcommand{\rname}[1]{\textsc{\footnotesize #1}}
\newcommand{\pure}[1]{|#1|}
\newcommand{\refl}{\text{refl}}
\newcommand{\letin}[3]{\ $\text{let }#1\text{ := }#2\text{ in }#3$\ }
\newcommand{\ind}[1]{\text{Ind}_{#1}}
\newcommand{\constr}{\text{Constr}}
\newcommand{\case}{\text{Case}}
\newcommand{\dcase}{\text{DCase}}
\newcommand{\fix}{\text{Fix }}
\newcommand{\sep}{\text{ | }}
\newcommand{\unit}{\text{unit}}
\newcommand{\new}{\text{new}}
\newcommand{\free}{\text{free}}
\newcommand{\get}{\text{get}}
\newcommand{\set}{\text{set}}
\newcommand{\session}{\text{session}}
\newcommand{\channel}{\text{channel}}
\newcommand{\open}{\text{open}}
\newcommand{\close}{\text{close}}
\newcommand{\send}{\text{send}}
\newcommand{\recv}{\text{recv}}
\newcommand{\SEND}{\texttt{SEND}}
\newcommand{\RECV}{\texttt{RECV}}
\newcommand{\END}{\texttt{END}}
\newcommand{\utype}{:_{\scriptscriptstyle U}}
\newcommand{\ltype}{:_{\scriptscriptstyle L}}
\newcommand{\stype}[1]{:_{#1}}
\newcommand{\step}{\leadsto}
\newcommand{\red}{\leadsto^*}
\newcommand{\pstep}{\leadsto_p}
\newcommand{\mrg}[3]{#1\ddagger#2\ddagger#3}
\newcommand{\erase}[1]{\llbracket #1 \rrbracket}
\newcommand{\lift}[1]{\llparenthesis #1 \rrparenthesis}
\newcommand{\lrangle}[1]{\langle #1 \rangle}
\newcommand{\ucons}{constructor_{\scriptscriptstyle U}}
\newcommand{\lcons}{constructor_{\scriptscriptstyle L}}
\newcommand{\scons}{constructor_{s}}
\newcommand{\inl}{\text{inl}}
\newcommand{\inr}{\text{inr}}


\title{The Calculus of Linear Constructions --- Technical Report}
\author{Qiancheng Fu}

\begin{document}
\maketitle
\tableofcontents

\section{Introduction}
This extended report is meant to accompany our paper of the same title. Here, we describe the meta-theory of CLC and CILC and their proofs in detail. All the results presented here have been formalized and proven correct in the Coq Proof Assistant.

\section{Syntax of CLC \texttt{(clc_ast.v)}}
\begin{figure}[H]
  \begin{grammar}
    <$i$> ::= 0 | 1 | 2 ... \phantom{* |} \hspace*{2.4em} universe levels

    <$s, t$> ::= $U$ | $L$ \phantom{| $x$} \hspace*{4.6em} sorts

    <$m, n, A, B, M$> ::= $U_i$ | $L_i$ | $x$ \hspace*{4em} expressions
    \indalt $(x :_s A) \rightarrow B$
    \indalt $(x :_s A) \multimap B$
    \indalt $\lambda x :_s A. n$
    \indalt $m\ n$
  \end{grammar}
\end{figure}

\section{Reduction and Equality of CLC \texttt{(clc_ast.v)}}
\begin{figure}[H]
  \begin{mathpar}
    \inferrule
    { m_1 \red n \\ m_2 \red n }
    { m_1 \equiv m_2 : A }
    \rname{Join}

    \inferrule
    {  }
    { (\lambda x \stype{s}A.m)\ n \step m[n/x] }
    \rname{Step-$\beta$}

    \inferrule
    { A \step A' }
    { \lambda x \stype{s}A.m \step \lambda x \stype{s}A' .m }
    \rname{Step-$\lambda$L}

    \inferrule
    { m \step m' }
    { \lambda x \stype{s}A.m \step \lambda x \stype{s}A.m' }
    \rname{Step-$\lambda$R}

    \inferrule
    { A \pstep A' }
    { (x \stype{s} A) \rightarrow B \step (x \stype{s} A') \rightarrow B }
    \rname{Step-L$\rightarrow$}

    \inferrule
    { B \pstep B' }
    { (x \stype{s} A) \rightarrow B \step (x \stype{s} A) \rightarrow B' }
    \rname{Step-R$\rightarrow$}

    \inferrule
    { A \pstep A' }
    { (x \stype{s} A) \multimap B \step (x \stype{s} A') \multimap B }
    \rname{Step-L$\multimap$}

    \inferrule
    { B \pstep B' }
    { (x \stype{s} A) \multimap B \step (x \stype{s} A) \multimap B' }
    \rname{Step-R$\multimap$}

    \inferrule
    { m \step m' }
    { m\ n \step m'\ n }
    \rname{Step-AppL}

    \inferrule
    { n \step n' }
    { m\ n \step m\ n' }
    \rname{Step-AppR}
  \end{mathpar}
  \label{red}
\end{figure}

\section{Confluence of CLC \texttt{(clc_confluence.v)}}

\subsection{Parallel Reduction}
To prove the confluence property of CLC, we employ the standard technique utilizing parallel reductions.

\begin{figure}[H]
  \begin{mathpar}
    \inferrule
    { }
    { x \pstep x }
    \rname{PStep-Var}

    \inferrule
    { }
    { s_i \pstep s_i }
    \rname{PStep-Sort}

    \inferrule
    { A \pstep A' \\ m \pstep m' }
    { \lambda x \stype{s} A.m \pstep \lambda x \stype{s} A'.m' }
    \rname{PStep-$\lambda$}

    \inferrule
    { m \pstep m' \\ n \pstep n' }
    { m\ n \pstep m'\ n' }
    \rname{PStep-App}

    \inferrule
    { m \pstep m' \\ n \pstep n'}
    { (\lambda x \stype{s} A.m)\ n \pstep m'[n'/x] }
    \rname{PStep-$\beta$}

    \inferrule
    { A \pstep A' \\ B \pstep B' }
    { (x \stype{s} A) \rightarrow B \pstep (x \stype{s} A') \rightarrow B' }
    \rname{PStep$\rightarrow$}

    \inferrule
    { A \pstep A' \\ B \pstep B' }
    { (x \stype{s} A) \multimap B \pstep (x \stype{s} A') \multimap B' }
    \rname{PStep$\multimap$}
  \end{mathpar}
  \label{pred}
\end{figure}

\subsection{Reduction Lemmas}
Here, we prove some simple lemmas concerning $\step$, $\red$ and substitution.

\begin{definition}
  For a term $m$ and a map $\sigma$ from variables to terms, let $m[\sigma]$ be the term obtained by applying $\sigma$ uniformly to all free variables in $m$.
\end{definition}

\begin{definition}
  For maps $\sigma, \tau$ from variables to terms, we say that $\sigma$ reduces to $\tau$ if for any variable $x$ there exists a reduction $(\sigma\ x) \red (\tau\ x)$. We write $\sigma \red \tau$ when it is clear from context that $\sigma, \tau$ are maps and not terms.
\end{definition}

\begin{lemma}\label{stepsubst}
  For terms $m, n$ and a map $\sigma$ from variables to terms, if there exist a step $m \step n$, then there exists a step $m[\sigma] \step n[\sigma]$.
\end{lemma}
\begin{proof}
  By induction on the derivation of $m \step n$.
\end{proof}

\begin{lemma}\label{redapp}
  For terms $m_1, m_2, n_1, n_2$, if these exists reductions $m_1 \red m_2$ and $n_1 \red n_2$, then there exists reduction $(m_1\ n_1) \red (m_2\ n_2)$.
\end{lemma}
\begin{proof}
  By transitivity of $\red$ and applying rules \rname{Step-AppL}, \rname{Step-AppR}.
\end{proof}

\begin{lemma}\label{redlam}
  For terms $A_1, A_2, m_1, m_2$ and sort $s$, if there exists reductions $A_1 \red A_2$ and $m_1 \red m_2$, then there exists reduction $\lambda x \stype{s}A_1.m_1 \red \lambda x \stype{s}A_2.m_2$.
\end{lemma}
\begin{proof}
  By transitivity of $\red$ and applying rules \rname{Step-$\lambda$L}, \rname{Step-$\lambda$R}.
\end{proof}

\begin{lemma}\label{redarrow}
  For terms $A_1, A_2, B_1, B_2$ and sort $s$, if there exists reductions $A_1 \red A_2$ and $B_1 \red B_2$, then there exists reduction $(x \stype{s}A_1) \rightarrow B_1 \red (x \stype{s} A_2) \rightarrow B_2$.
\end{lemma}
\begin{proof}
  By transitivity of $\red$ and applying rules \rname{Step-L$\rightarrow$}, \rname{Step-R$\rightarrow$}.
\end{proof}

\begin{lemma}\label{redlolli}
  For terms $A_1, A_2, B_1, B_2$ and sort $s$, if there exists reductions $A_1 \red A_2$ and $B_1 \red B_2$, then there exists reduction $(x \stype{s}A_1) \multimap B_1 \red (x \stype{s} A_2) \multimap B_2$.
\end{lemma}
\begin{proof}
  By transitivity of $\red$ and applying rules \rname{Step-L$\multimap$}, \rname{Step-R$\multimap$}.
\end{proof}

\begin{lemma}\label{redsubst}
  For terms $m, n$ and a map $\sigma$ from variables to terms, if there exist a reduction $m \red n$, then there exists a reduction $m[\sigma] \red n[\sigma]$.
\end{lemma}
\begin{proof}
  By induction on the derivation of $\red$, the transitivity of $\red$ and Lemma \ref{stepsubst}.
\end{proof}

\begin{lemma}\label{redcompat}
  For maps $\sigma, \tau$ from variables to terms, if there is a map reduction $\sigma \red \tau$, then for any term $m$ these is a reduction $m[\sigma] \red m[\tau]$.
\end{lemma}
\begin{proof}
  By induction on the structure of $m$, applying Lemmas \ref{redapp}, \ref{redlam}, \ref{redarrow}, \ref{redlolli}.
\end{proof}

\subsection{Equality Lemmas}
Here, we prove some simple lemmas concerning $\red$, $\equiv$ and substitution.

\begin{definition}
  For maps $\sigma, \tau$ from variables to terms, we say that $\sigma$ is equal to $\tau$ if for any variable $x$ there exists an equality $(\sigma\ x) \equiv (\tau\ x)$. We write $\sigma \equiv \tau$ when it is clear from context that $\sigma, \tau$ are maps and not terms.
\end{definition}

\begin{lemma}\label{convhom}
  For any map $f$ from terms to terms, if for any terms $m, n$ such that $m \step n$ implies $f\ m \equiv f\ n$, then for any terms $m, n$ equality $m \equiv n$ implies $f\ m \equiv f\ n$.
\end{lemma}
\begin{proof}
  By the properties of the transitive reflexive closure $\red$ and that $\equiv$ is an equivalence relation.
\end{proof}

\begin{lemma}\label{convapp}
  For terms $m_1, m_2, n_1, n_2$, if there exists equalities $m_1 \equiv m_2$ and $n_1 \equiv n_2$, then there exists equality $(m_1\ n_1) \equiv (m_2\ n_2)$.
\end{lemma}
\begin{proof}
  By transitivity of $\equiv$ and applying rules \rname{Join}, \rname{Step-AppL}, \rname{Step-AppR}.
\end{proof}

\begin{lemma}\label{convlam}
  For terms $A_1, A_2, m_1, m_2$ and sort $s$, if there exists equalities $A_1 \equiv A_2$ and $m_1 \equiv m_2$, then there exists equality $\lambda x \stype{s} A_1.m_1 \equiv \lambda x \stype{s} A_2.m_2$.
\end{lemma}
\begin{proof}
  By transitivity of $\equiv$ and applying rules \rname{Join}, \rname{Step-$\lambda$L}, \rname{Step-$\lambda$R}.
\end{proof}

\begin{lemma}\label{convarrow}
  For terms $A_1, A_2, B_1, B_2$ and sort $s$, if there exists equalities $A_1 \equiv A_2$ and $B_1 \equiv B_2$, then there exists equality $(x \stype{s} A_1) \rightarrow B_1 \equiv (x \stype{s} A_2) \rightarrow B_2$.
\end{lemma}
\begin{proof}
  By transitivity of $\equiv$ and applying rules \rname{Join}, \rname{Step-L$\rightarrow$}, \rname{Step-R$\rightarrow$}.
\end{proof}

\begin{lemma}\label{convlolli}
  For terms $A_1, A_2, B_1, B_2$ and sort $s$, if there exists equalities $A_1 \equiv A_2$ and $B_1 \equiv B_2$, then there exists equality $(x \stype{s} A_1) \multimap B_1 \equiv (x \stype{s} A_2) \multimap B_2$.
\end{lemma}
\begin{proof}
  By transitivity of $\equiv$ and applying rules \rname{Join}, \rname{Step-L$\multimap$}, \rname{Step-R$\multimap$}.
\end{proof}

\begin{lemma}\label{convsubst}
  For terms $m, n$ and map $\sigma$ from variables to terms, if there is equality $m \equiv n$, then there is equality $m[\sigma] \equiv n[\sigma]$.
\end{lemma}
\begin{proof}
  By Lemmas \ref{convhom} and \ref{stepsubst}.
\end{proof}

\begin{lemma}\label{convcompat}
  For maps $\sigma, \tau$ from variables to terms and term $m$, if these is map equality $\sigma \equiv \tau$, then there is equality $m[\sigma] \equiv m[\tau]$.
\end{lemma}
\begin{proof}
  By induction on the structure of $m$, applying Lemmas \ref{convapp}, \ref{convlam}, \ref{convarrow}, \ref{convlolli}.
\end{proof}

\begin{lemma}
  For terms $m_1, m_2, n$, if there is equality $m_1 \equiv m_2$, then there is equality $n[m_1/x] \equiv n[m_2/x]$ for any variable $x \in FV(n)$.
\end{lemma}
\begin{proof}
  This is a special case of Lemma \ref{convcompat} where $\sigma$ maps $x$ to $m_1$ and $\tau$ maps $x$ to $m_2$.
\end{proof}

\subsection{Parallel Reduction Lemmas}

\begin{definition}\label{psstep}
  For maps $\sigma, \tau$ from variables to terms, we say $\sigma$ parallel reduces to $\tau$ if for any variable $x$ there exists a parallel reduction $(\sigma\ x) \pstep (\tau\ x)$. We write $\sigma \pstep \tau$ when it is clear from context that $\sigma, \tau$ are maps and not terms.
\end{definition}

\begin{lemma}\label{psteprefl}
  For any term $m$, there exists a reflexive parallel reduction $m \pstep m$.
\end{lemma}
\begin{proof}
  By induction on the structure of $m$.
\end{proof}

\begin{lemma}
  For any map $\sigma$ from variables to terms, there exists a reflexive parallel map reduction $\sigma \pstep \sigma$.
\end{lemma}
\begin{proof}
  By Definition \ref{psstep} and Lemma \ref{psteprefl}.
\end{proof}

\begin{lemma}\label{steppstep}
  For any terms $m, n$, if there exists step $m \step n$, then there exists a parallel reduction $m \pstep n$.
\end{lemma}
\begin{proof}
  By induction on the derivation of $m \step n$ and Lemma \ref{psteprefl}.
\end{proof}

\begin{lemma}\label{pstepred}
  For terms $m, n$, if there exists parallel reduction $m \pstep n$, then there exists a reduction $m \red n$.
\end{lemma}
\begin{proof}
  By induction on the derivation of $m \pstep n$, utilizing the transitive property of $\red$ and Lemmas \ref{redapp}, \ref{redlam}, \ref{redarrow}, \ref{redlolli}, \ref{redsubst}, \ref{redcompat}.
\end{proof}

\begin{lemma}
  For terms $m, n$ and map $\sigma$ from variables to terms, if there exists parallel reduction $m \pstep n$, there exists parallel reduction $m[\sigma] \pstep n[\sigma]$.
\end{lemma}
\begin{proof}
  By induction on the derivation of $m \pstep n$ and Lemma \ref{psteprefl}.
\end{proof}

\begin{lemma}\label{pstepcompat}
  For terms $m, n$ and maps $\sigma, \tau$ from variables to terms, if there exists parallel reduction $m \pstep n$ and parallel map reduction $\sigma \pstep \tau$, there exists parallel reduction $m[\sigma] \pstep n[\tau]$.
\end{lemma}
\begin{proof}
  By induction on the derivation of $m \pstep n$.
\end{proof}

\begin{lemma}
  For terms $m_1, m_2, n$, if there is parallel reduction $m_1 \pstep m_2$, then there is parallel reduction $n[m_1/x] \pstep n[m_2/x]$ for any variable $x \in FV(n)$.
\end{lemma}
\begin{proof}
  By Lemma \ref{psteprefl}, this is a special case of Lemma \ref{pstepcompat} where $\sigma$ maps $x$ to $m_1$ and $\tau$ maps $x$ to $m_2$.
\end{proof}

\subsection{Confluence Theorem}
We first show that $\pstep$ satisfies the diamond property. Using the diamond property, we ultimately prove the confluence theorem.

\begin{lemma}\label{diamond}
  CLC term reduction has the diamond property. For terms $m, m_1, m_2$, if there are parallel reductions $m \pstep m_1$ and $m \pstep m_2$, then there exists term $m'$ such that $m_1 \pstep m'$ and $m_2 \pstep m'$.
\end{lemma}
\begin{proof}
  By induction on the derivation of $m \pstep m_1$. Each case in the induction specializes $m$ appearing in $m \pstep m_2$, allowing one to invert its derivation in a syntax directed way and apply the induction hypothesis. The difficult cases are due to \rname{PStep-$\beta$} as it concerns substitution, so Lemma \ref{pstepcompat} is used to push these cases through.
\end{proof}

\begin{lemma}\label{strip}
  Strip lemma. For terms $m, m_1, m_2$, if there is parallel reduction $m \pstep m_1$ and reduction $m \red m_2$, then there exists term $m'$ such that $m_1 \red m'$ and $m_2 \pstep m'$.
\end{lemma}
\begin{proof}
  By induction on the derivation of $m \pstep m_1$, utilizing transitivity of $\red$ and Lemmas \ref{steppstep}, \ref{pstepred}, \ref{diamond}.
\end{proof}

\begin{theorem}
  CLC term reduction is confluent. For terms $m, m_1, m_2$, if there are reductions $m \red m_1$ and $m \red m_2$, then there exists term $m'$ such that $m_1 \red m'$ and $m_2 \red m'$.
\end{theorem}
\begin{proof}
  By induction on the derivation of $m \red m_1$, utilizing transitivity of $\red$ and Lemmas \ref{steppstep}, \ref{pstepred}, \ref{strip}.
\end{proof}

\subsection{Corollaries of Confluence}
The following results are all corollaries of confluence, proven using a combination of induction, transitivity and confluence. These corollaries allow us to refute false reductions and equalities in future proofs.

\begin{corollary}\label{redsortinv}
  For a universe $s_i$ and term $m$, if there is reduction $s_i \red m$, then $m = s_i$.
\end{corollary}

\begin{corollary}\label{redvarinv}
  For variable $x$ and term $m$, if there is reduction $x \red m$, then $m = x$.
\end{corollary}

\begin{corollary}\label{redarrowinv}
  For terms $A, B, m$ and sort $s$, if there is reduction $(x \stype{s} A) \rightarrow B \red m$, then there exists $A', B'$ such that there are reductions $A \red A'$, $B \red B'$ and $m = (x \stype{s} A') \rightarrow B'$.
\end{corollary}

\begin{corollary}\label{redlolliinv}
  For terms $A, B, m$ and sort $s$, if there is reduction $(x \stype{s} A) \multimap B \red m$, then there exists $A', B'$ such that there are reductions $A \red A'$, $B \red B'$ and $m = (x \stype{s} A') \multimap B'$.
\end{corollary}

\begin{corollary}\label{redlaminv}
  For terms $A, m, n$ and sort $s$, if there is reduction $\lambda x \stype{s} A.m \red n$, then there exists $A',m'$ such that there are reductions $A \red A'$, $m \red m'$ and $n = \lambda x \stype{s} A'.m'$.
\end{corollary}

\begin{corollary}\label{sortinj}
  For sorts $s, t$ and levels $i, j$, if there is equality $s_i \equiv t_j$, then there is $s = t$ and $i = j$.
\end{corollary}

\begin{corollary}\label{arrowinj}
  For terms $A_1, A_2, B_1, B_2$ and sorts $s, t$, if there is equality $(x \stype{s} A_1) \rightarrow B_1 \equiv (x \stype{t} A_2) \rightarrow B_2$, then there are equalities $A_1 \equiv A_2$, $B_1 \equiv B_2$ and $s = t$.
\end{corollary}

\begin{corollary}\label{lolliinj}
  For terms $A_1, A_2, B_1, B_2$ and sorts $s, t$, if there is equality $(x \stype{s} A_1) \multimap B_1 \equiv (x \stype{t} A_2) \multimap B_2$, then there are equalities $A_1 \equiv A_2$, $B_1 \equiv B_2$ and $s = t$.
\end{corollary}

\section{Context of CLC \texttt{(clc_context.v)}}
Contexts of CLC are of the form $x_1 \stype{s_1} A_1, x_2 \stype{s_2} A_2, ... x_k \stype{s_k} A_k$ where each free variable $x_i$ is assigned a type $A_i$ and sort $s_i$. Contexts will be referred to by meta variables $\Gamma$ and $\Delta$.

\begin{figure}[h]
  \begin{mathpar}
    \inferrule
    { }
    { \epsilon \vdash }
    \rname{Wf-$\epsilon$}

    \inferrule
    { \Gamma\ \vdash \\
      \overline{\Gamma} \vdash A : U_i }
    { \Gamma, x \utype A \vdash }
    \rname{Wf-U}

    \inferrule
    { \Gamma\ \vdash \\
      \overline{\Gamma} \vdash A : L_i }
    { \Gamma, x \ltype A\ \vdash }
    \rname{Wf-L}
    \\

    \inferrule
    { }
    { \pure{\epsilon} }
    \rname{Pure-$\epsilon$}

    \inferrule
    { \pure{\Gamma} \\
      \Gamma \vdash A : U_i }
    { \pure{\Gamma, x \utype A} }
    \rname{Pure-U}
    \\

    \inferrule
    { }
    { \mrg{\epsilon}{\epsilon}{\epsilon} }
    \rname{Merge-$\epsilon$}

    \inferrule
    { \mrg{\Gamma_1}{\Gamma_2}{\Gamma} }
    { \mrg{\Gamma_1, x \utype A}
      {\Gamma_2, x \utype A}
      {\Gamma, x \utype A} }
    \rname{Merge-U}
    \\

    \inferrule
    { \mrg{\Gamma_1}{\Gamma_2}{\Gamma} \\
      x \notin \Gamma_2 }
    { \mrg{\Gamma_1, x \ltype A}
      {\Gamma_2}
      {\Gamma, x \ltype A} }
    \rname{Merge-L1}

    \inferrule
    { \mrg{\Gamma_1}{\Gamma_2}{\Gamma} \\
      x \notin \Gamma_1 }
    { \mrg{\Gamma_1}
      {\Gamma_2, x \ltype A}
      {\Gamma, x \ltype A} }
    \rname{Merge-L2}
  \end{mathpar}
  \label{structural}
\end{figure}

\subsection{Merge Lemmas}\label{mergeprop}
Since weakening and contraction rules will not be allowed on restricted variables, it is necessary to have lemmas that enable the manipulation of contexts.

\begin{lemma}\label{mergesym}
  For contexts $\Gamma_1, \Gamma_2, \Gamma$, if there is $\mrg{\Gamma_1}{\Gamma_2}{\Gamma}$, then there is $\mrg{\Gamma_2}{\Gamma_1}{\Gamma}$.
\end{lemma}
\begin{proof}
  By induction on the derivation of $\mrg{\Gamma_1}{\Gamma_2}{\Gamma}$.
\end{proof}

\begin{lemma}\label{mergepure}
  For any context $\Gamma$, if there is $\pure{\Gamma}$, then there is $\mrg{\Gamma}{\Gamma}{\Gamma}$.
\end{lemma}
\begin{proof}
  By induction on the derivation of $\pure{\Gamma}$.
\end{proof}

\begin{lemma}\label{mergere1}
  For any context $\Gamma$, there is $\mrg{\overline{\Gamma}}{\Gamma}{\Gamma}$.
\end{lemma}
\begin{proof}
  By induction on the structure of $\Gamma$.
\end{proof}

\begin{lemma}\label{mergere2}
  For any context $\Gamma$, there is $\mrg{\Gamma}{\overline{\Gamma}}{\Gamma}$.
\end{lemma}
\begin{proof}
  By induction on the structure of $\Gamma$.
\end{proof}

\begin{lemma}\label{mergepureinv}
  For contexts $\Gamma_1, \Gamma_2, \Gamma$, if there is $\mrg{\Gamma_1}{\Gamma_2}{\Gamma}$ and $\pure{\Gamma}$, then there is $\pure{\Gamma_1}$ and $\pure{\Gamma_2}$.
\end{lemma}
\begin{proof}
  By induction on the derivation of $\mrg{\Gamma_1}{\Gamma_2}{\Gamma}$.
\end{proof}

\begin{lemma}\label{mergepure1}
  For contexts $\Gamma_1, \Gamma_2, \Gamma$ , if there is $\mrg{\Gamma_1}{\Gamma_2}{\Gamma}$ and $\pure{\Gamma_1}$, then there is $\Gamma = \Gamma_2$.
\end{lemma}
\begin{proof}
  By induction on the derivation of $\mrg{\Gamma_1}{\Gamma_2}{\Gamma}$.
\end{proof}

\begin{lemma}\label{mergepure2}
  For contexts $\Gamma_1, \Gamma_2, \Gamma$ , if there is $\mrg{\Gamma_1}{\Gamma_2}{\Gamma}$ and $\pure{\Gamma_2}$, then there is $\Gamma = \Gamma_1$.
\end{lemma}
\begin{proof}
  By induction on the derivation of $\mrg{\Gamma_1}{\Gamma_2}{\Gamma}$.
\end{proof}

\begin{lemma}\label{mergepurepure}
  For contexts $\Gamma_1, \Gamma_2, \Gamma$, if there is $\mrg{\Gamma_1}{\Gamma_2}{\Gamma}$, and also $\pure{\Gamma_1}$, $\pure{\Gamma_2}$, then there is $\pure{\Gamma}$.
\end{lemma}
\begin{proof}
  By induction on the derivation of $\mrg{\Gamma_1}{\Gamma_2}{\Gamma}$.
\end{proof}

\begin{lemma}\label{mergepureeq}
  For contexts $\Gamma_1, \Gamma_2, \Gamma$, if there is $\mrg{\Gamma_1}{\Gamma_2}{\Gamma}$, and also $\pure{\Gamma_1}$, $\pure{\Gamma_2}$, then there is $\Gamma_1 = \Gamma_2$.
\end{lemma}
\begin{proof}
  By induction on the derivation of $\mrg{\Gamma_1}{\Gamma_2}{\Gamma}$.
\end{proof}

\begin{lemma}\label{mergerere}
  For contexts $\Gamma_1, \Gamma_2, \Gamma$, if there is $\mrg{\Gamma_1}{\Gamma_2}{\Gamma}$, then there is $\overline{\Gamma_1} = \overline{\Gamma}$ and $\overline{\Gamma_2} = \overline{\Gamma}$.
\end{lemma}
\begin{proof}
  By induction on the derivation of $\mrg{\Gamma_1}{\Gamma_2}{\Gamma}$.
\end{proof}

\begin{lemma}\label{mergererere}
  For any context $\Gamma$, there is $\mrg{\overline{\Gamma}}{\overline{\Gamma}}{\overline{\Gamma}}$.
\end{lemma}
\begin{proof}
  By induction on the structure of $\Gamma$.
\end{proof}

\subsection{Restriction and Purity Lemmas}

\begin{lemma}\label{rere}
  For any context $\Gamma$, there is $\overline{\Gamma} = \overline{\overline{\Gamma}}$.
\end{lemma}
\begin{proof}
  By induction on the structure of $\Gamma$.
\end{proof}

\begin{lemma}\label{purere}
  For any context $\Gamma$, if there is $\pure{\Gamma}$, then there is $\Gamma = \overline{\Gamma}$.
\end{lemma}
\begin{proof}
  By induction on the structure of $\Gamma$.
\end{proof}

\begin{lemma}\label{repure}
  For any context $\Gamma$ , there is $\pure{\overline{\Gamma}}$.
\end{lemma}
\begin{proof}
  By induction on the structure of $\Gamma$.
\end{proof}

\begin{lemma}\label{hasure}
  For any context $\Gamma$, variable $x$ and type $A$, if there is $x \utype A \in \Gamma$, then there is $x \utype A \in \overline{\Gamma}$.
\end{lemma}
\begin{proof}
  By induction on the derivation of $x \utype A \in \Gamma$.
\end{proof}

\begin{lemma}\label{haslre}
  For any context $\Gamma$, variable $x$ and type $A$, there is $x \ltype A \notin \overline{\Gamma}$.
\end{lemma}
\begin{proof}
  By induction on the structure of $\Gamma$.
\end{proof}

\begin{lemma}\label{mergesplit1}
  For contexts $\Gamma_1, \Gamma_2, \Gamma, \Delta_1, \Delta_2$, if there is $\mrg{\Gamma_1}{\Gamma_2}{\Gamma}$ and $\mrg{\Delta_1}{\Delta_2}{\Gamma_1}$, then there exists $\Delta$ such that $\mrg{\Delta_1}{\Gamma_2}{\Delta}$ and $\mrg{\Delta}{\Delta_2}{\Gamma}$.
\end{lemma}
\begin{proof}
  By induction on the derivation of $\mrg{\Gamma_1}{\Gamma_2}{\Gamma}$.
\end{proof}

\begin{lemma}
  For contexts $\Gamma_1, \Gamma_2, \Gamma, \Delta_1, \Delta_2$, if there is $\mrg{\Gamma_1}{\Gamma_2}{\Gamma}$ and $\mrg{\Delta_1}{\Delta_2}{\Gamma_1}$, then there exists $\Delta$ such that $\mrg{\Delta_2}{\Gamma_2}{\Delta}$ and $\mrg{\Delta_1}{\Delta}{\Gamma}$.
\end{lemma}
\begin{proof}
  By induction on the derivation of $\mrg{\Gamma_1}{\Gamma_2}{\Gamma}$.
\end{proof}

\section{Subtyping of CLC \texttt{(clc_subtype.v)}}

The cumulativity relation ($\preceq$) is the smallest binary relation over terms such that
\begin{enumerate}
  \item $\preceq$ is a partial order with respect to equality.
        \begin{enumerate}
          \item If $A \equiv B$, then $A \preceq B$.
          \item If $A \preceq B$ and $B \preceq A$, then $A \equiv B$.
          \item If $A \preceq B$ and $B \preceq C$, then $A \preceq B$.
        \end{enumerate}
  \item $U_0 \preceq U_1 \preceq U_2 \preceq \cdots$
  \item $L_0 \preceq L_1 \preceq L_2 \preceq \cdots$
  \item If $A_1 \equiv A_2$ and $B_1 \preceq B_2$, \\ then
        $(x \stype{s} A_1) \rightarrow B_1 \preceq (x \stype{s} A_2) \rightarrow B_2$
  \item If $A_1 \equiv A_2$ and $B_1 \preceq B_2$, \\ then
        $(x \stype{s} A_1) \multimap B_1 \preceq (x \stype{s} A_2) \multimap B_2$
\end{enumerate}

\noindent
Here, we give an inductive definition of the cumulativity relation ($\preceq$) that is suitable for writing proofs.

\begin{mathpar}
  \inferrule
  { }
  { A \prec A }
  \rname{$\prec$-Refl}

  \inferrule
  { i_1 \leq i_2 }
  { s_{i_1} \prec s_{i_2} }
  \rname{$\prec$-Sort}

  \inferrule
  { B_1 \prec B_2 }
  { (x \stype{s} A) \rightarrow B_1 \prec (x \stype{s} A) \rightarrow B_2 }
  \rname{$\prec$$\rightarrow$}

  \inferrule
  { B_1 \prec B_2 }
  { (x \stype{s} A) \multimap B_1 \prec (x \stype{s} A) \multimap B_2 }
  \rname{$\prec$$\multimap$}

  \inferrule
  { A' \prec B' \\ A \equiv A' \\ B \equiv B' }
  { A \preceq B }
  \rname{$\prec$-$\preceq$}
\end{mathpar}

\subsection{Subtyping Lemmas}

\begin{lemma}\label{sub1sub}
  For terms $A, B$, if there is $A \prec B$, then there is $A \preceq B$.
\end{lemma}
\begin{proof}
  By \rname{$\prec$-$\preceq$} and the reflexivity of equality $\equiv$.
\end{proof}

\begin{lemma}\label{sub1conv}
  For terms $A, B, C$, if there is $A \prec B$ and $B \equiv C$, then there is $A \preceq C$.
\end{lemma}
\begin{proof}
  By \rname{$\prec$-$\preceq$} and the transitivity of equality $\equiv$.
\end{proof}

\begin{lemma}\label{convsub1}
  For terms $A, B, C$, if there is $A \equiv B$ and $B \prec C$, then there is $A \preceq C$.
\end{lemma}
\begin{proof}
  By \rname{$\prec$-$\preceq$} and the transitivity of equality $\equiv$.
\end{proof}

\begin{lemma}\label{convsub}
  For terms $A, B$, if there is $A \equiv B$, then there is $A \preceq B$.
\end{lemma}
\begin{proof}
  By Lemma \ref{convsub1} and \rname{$\prec$-Refl}.
\end{proof}

\begin{lemma}\label{subrefl}
  For term $A$, there is $A \preceq A$.
\end{lemma}
\begin{proof}
  By Lemma \ref{sub1sub} and \rname{$\prec$-Refl}.
\end{proof}

\begin{lemma}\label{subprop}
  For natural numbers $i, j$ and sort $s$ such that $i \leq j$, there is $s_i \preceq s_j$.
\end{lemma}
\begin{proof}
  By Lemma \ref{sub1sub} and \rname{$\prec$-Sort}.
\end{proof}

\begin{lemma}\label{sub1trans}
  For terms $A, B, C, D$, if there is $A \prec B$, $B \equiv C$ and $C \prec D$, then there is $A \preceq D$.
\end{lemma}
\begin{proof}
  By induction on the derivation of $A \prec B$, definition of $\prec$ and Lemmas \ref{sub1sub}, \ref{sub1conv}, \ref{convsub1}.
\end{proof}

\begin{lemma}
  For terms $A, B, C$, if there is $A \preceq B$ and $B \preceq C$, then there is $A \preceq C$.
\end{lemma}
\begin{proof}
  By transitivity of $\equiv$, rule \rname{$\prec$-$\preceq$} and Lemma \ref{sub1trans}.
\end{proof}

\begin{lemma}
  For sorts $s, t$ and natural numbers $i, j$, if there is $s_i \preceq t_j$, then there is $s = t$ and $i \leq j$.
\end{lemma}
\begin{proof}
  By transitivity of $\equiv$ and Corollary \ref{sortinj}.
\end{proof}

\begin{lemma}
  For terms $A_1, A_2, B_1, B_2$ and sorts $s, t$, if there is $(x \stype{s} A_1) \rightarrow B_1 \preceq (x \stype{t} A_2) \rightarrow B_2$, then there is $A_1 \equiv A_2$ and $B_1 \preceq B_2$ and $s = t$.
\end{lemma}
\begin{proof}
  By transitivity of $\equiv$ and Corollary \ref{arrowinj}.
\end{proof}

\begin{lemma}
  For terms $A_1, A_2, B_1, B_2$ and sorts $s, t$, if there is $(x \stype{s} A_1) \multimap B_1 \preceq (x \stype{t} A_2) \multimap B_2$, then there is $A_1 \equiv A_2$ and $B_1 \preceq B_2$ and $s = t$.
\end{lemma}
\begin{proof}
  By transitivity of $\equiv$ and Corollary \ref{lolliinj}.
\end{proof}

\begin{lemma}\label{sub1subst}
  For terms $A, B$  and map $\sigma$ from variables to terms, if there is $A \prec B$, then there is $A[\sigma] \prec B[\sigma]$.
\end{lemma}
\begin{proof}
  By induction on the derivation of $A \prec B$ and the definition of $\prec$.
\end{proof}

\begin{lemma}
  For terms $A, B$ and map $\sigma$ from variables to terms, if there is $A \preceq B$, then there is $A[\sigma] \preceq B[\sigma]$.
\end{lemma}
\begin{proof}
  By rule \rname{$\prec$-$\preceq$} and Lemmas \ref{convsubst}, \ref{sub1subst}.
\end{proof}

\section{Typing of CLC \texttt{(clc_typing.v)}}

The following rules define well-formed contexts.
\begin{mathpar}
  \inferrule
  { }
  { \epsilon \vdash }
  \rname{$\epsilon$-Ok}

  \inferrule
  { \Gamma \vdash \\
    \overline{\Gamma} \vdash A : U_i }
  { \Gamma, x \utype A \vdash }
  \rname{U-Ok}

  \inferrule
  { \Gamma \vdash \\
    \overline{\Gamma} \vdash A : L_i }
  { \Gamma, x \ltype A \vdash }
  \rname{L-Ok}
\end{mathpar}

\noindent
The typing rules of CLC are presented below.
\begin{mathpar}
  \inferrule
  { \pure{\Gamma} }
  { \Gamma \vdash s_i : U_{i+1} }
  \rname{Sort-Axiom}

  \inferrule
  { \pure{\Gamma} \\
    \Gamma \vdash A : U_i \\
    \Gamma, x \utype A \vdash B : s_i }
  { \Gamma \vdash (x \utype A) \rightarrow B : U_i }
  \rname{U$\rightarrow$}

  \inferrule
  { \pure{\Gamma} \\
    \Gamma \vdash A : L_i \\
    \Gamma \vdash B : s_i \\
    x \notin \Gamma }
  { \Gamma \vdash (x \ltype A) \rightarrow B : U_i }
  \rname{L$\rightarrow$}

  \inferrule
  { \pure{\Gamma} \\
    \Gamma \vdash A : U_i \\
    \Gamma, x \utype A \vdash B : s_i }
  { \Gamma \vdash (x \utype A) \multimap B : L_i }
  \rname{U$\multimap$}

  \inferrule
  { \pure{\Gamma} \\
    \Gamma \vdash A : L_i \\
    \Gamma \vdash B : s_i \\
    x \notin \Gamma }
  { \Gamma \vdash (x \ltype A) \multimap B : L_i }
  \rname{L$\multimap$}

  \inferrule
  { \pure{\Gamma_1, \Gamma_2} }
  { \Gamma_1, x \utype A, \Gamma_2 \vdash x : A }
  \rname{U-Var}

  \inferrule
  { \pure{\Gamma_1, \Gamma_2} }
  { \Gamma_1, x \ltype A, \Gamma_2 \vdash x : A }
  \rname{L-Var}

  \inferrule
  { \pure{\Gamma} \\
    \Gamma \vdash (x \stype{s} A) \rightarrow B : t_i \\
    \Gamma, x \stype{s} A \vdash n : B }
  { \Gamma \vdash \lambda x \stype{s} A . n : (x \stype{s} A) \rightarrow B }
  \rname{$\lambda$$\rightarrow$}

  \inferrule
  { \overline{\Gamma} \vdash (x \stype{s} A) \multimap B : t_i \\
    \Gamma, x \stype{s} A \vdash n : B }
  { \Gamma \vdash \lambda x \stype{s} A . n : (x \stype{s} A) \multimap B }
  \rname{$\lambda$$\multimap$}

  \inferrule
  { \Gamma_1 \vdash m : (x \utype A) \rightarrow B \\
    \pure{\Gamma_2} \\
    \Gamma_2 \vdash n : A \\
    \mrg{\Gamma_1}{\Gamma_2}{\Gamma} }
  { \Gamma \vdash m\ n : B[n/x] }
  \rname{App-U$\rightarrow$}

  \inferrule
  { \Gamma_1 \vdash m : (x \ltype A) \rightarrow B \\
    \Gamma_2 \vdash n : A \\
    \mrg{\Gamma_1}{\Gamma_2}{\Gamma} }
  { \Gamma \vdash m\ n : B[n/x] }
  \rname{App-L$\rightarrow$}
\end{mathpar}

\begin{mathpar}
  \inferrule
  { \Gamma_1 \vdash m : (x \utype A) \multimap B \\
    \pure{\Gamma_2} \\
    \Gamma_2 \vdash n : A \\
    \mrg{\Gamma_1}{\Gamma_2}{\Gamma} }
  { \Gamma \vdash m\ n : B[n/x] }
  \rname{App-U$\multimap$}

  \inferrule
  { \Gamma_1 \vdash m : (x \ltype A) \multimap B \\
    \Gamma_2 \vdash n : A \\
    \mrg{\Gamma_1}{\Gamma_2}{\Gamma} }
  { \Gamma \vdash m\ n : B[n/x] }
  \rname{App-L$\multimap$}

  \inferrule
  { \Gamma \vdash m : A \\
    \overline{\Gamma} \vdash B : s_i \\ A \preceq B }
  { \Gamma \vdash m : B }
  \rname{Conversion}
\end{mathpar}

\section{Inversion Lemmas of CLC \texttt{(clc_inversion.v)}}

\begin{lemma}\label{uarrowinv}
  For any context $\Gamma$ and terms $A, B, s$, if there is $\Gamma \vdash (x \utype A) \rightarrow B : s$, then there exists sort $t$ and natural number $i$ such that $\Gamma \vdash A : U_i$ and $\Gamma, x \utype A \vdash B : t_i$.
\end{lemma}
\begin{proof}
  By induction on the derivation of $\Gamma \vdash (x \utype A) \rightarrow B : s$.
\end{proof}

\begin{lemma}\label{larrowinv}
  For any context $\Gamma$ and terms $A, B, s$, if there is $\Gamma \vdash (x \ltype A)\rightarrow B : s$, then there exists sort $t$ and natural number $i$ such that $\Gamma \vdash A : L_i$ and $\Gamma \vdash B : t_i$.
\end{lemma}
\begin{proof}
  By induction on the derivation of $\Gamma \vdash (x \ltype A) \rightarrow B : s$.
\end{proof}

\begin{lemma}\label{ulolliinv}
  For any context $\Gamma$ and terms $A, B, s$, if there is $\Gamma \vdash (x \utype A) \multimap B : s$, then there exists sort $t$ and natural number $i$ such that $\Gamma \vdash A : U_i$ and $\Gamma, x \utype A \vdash B : t_i$.
\end{lemma}
\begin{proof}
  By induction on the derivation of $\Gamma \vdash (x \utype A) \multimap B : s$.
\end{proof}

\begin{lemma}\label{llolliinv}
  For any context $\Gamma$ and terms $A, B, s$, if there is $\Gamma \vdash (x \ltype A)\multimap B : s$, then there exists sort $t$ and natural number $i$ such that $\Gamma \vdash A : L_i$ and $\Gamma \vdash B : t_i$.
\end{lemma}
\begin{proof}
  By induction on the derivation of $\Gamma \vdash (x \ltype A) \multimap B : s$.
\end{proof}

\begin{lemma}\label{arrowlaminvx}
  For any context $\Gamma$, terms $A, n, C$ and sort $s$, if there is $\Gamma \vdash \lambda x \stype{s} A . n : C$, then for all terms $A', B$, sorts $s', t$ and natural number $i$ such that $C \preceq (x \stype{s'} A') \rightarrow B$ and $\overline{\Gamma, x \stype{s'} A'} \vdash B : t_i$, there is $\Gamma, x \stype{s'} A' \vdash n : B$.
\end{lemma}
\begin{proof}
  By induction on the derivation of $\Gamma \vdash \lambda x \stype{s} A . n : C$ and Lemmas \ref{uarrowinv}, \ref{larrowinv}.
\end{proof}

\begin{lemma}\label{lollilaminvx}
  For any context $\Gamma$, terms $A, n, C$ and sort $s$, if there is $\Gamma \vdash \lambda x \stype{s} A . n : C$, then for all terms $A', B$, sorts $s', t$ and natural number $i$ such that $C \preceq (x \stype{s'} A') \multimap B$ and $\overline{\Gamma, x \stype{s'} A'} \vdash B : t_i$, there is $\Gamma, x \stype{s'} A' \vdash n : B$.
\end{lemma}
\begin{proof}
  By induction on the derivation of $\Gamma \vdash \lambda x \stype{s} A . n : C$ and Lemmas \ref{ulolliinv}, \ref{llolliinv}.
\end{proof}

\begin{lemma}\label{arrowlaminv}
  For any context $\Gamma$, terms $A, A', B, n$, sorts $s, s', t$ and natural number $i$, if there is $\overline{\Gamma} \vdash (x \stype{s'} A') \rightarrow B : t_i$ and $\Gamma \vdash \lambda x \stype{s} A . n : (x \stype{s'} A') \rightarrow B$, then there is $\Gamma, x \stype{s'} A' \vdash n : B$.
\end{lemma}
\begin{proof}
  Direct consequence of Lemmas \ref{uarrowinv}, \ref{larrowinv} and \ref{arrowlaminvx}.
\end{proof}

\begin{lemma}\label{lollilaminv}
  For any context $\Gamma$, terms $A, A', B, n$, sorts $s, s', t$ and natural number $i$, if there is $\overline{\Gamma} \vdash (x \stype{s'} A') \multimap B : t_i$ and $\Gamma \vdash \lambda x \stype{s} A . n : (x \stype{s'} A') \multimap B$, then there is $\Gamma, x \stype{s'} A' \vdash n : B$.
\end{lemma}
\begin{proof}
  Direct consequence of Lemmas \ref{ulolliinv}, \ref{llolliinv} and \ref{lollilaminvx}.
\end{proof}

\section{Weakening Lemmas of CLC \texttt{(clc_weakening.v)}}

Weakening for non-linear types is admissible in CLC. To prove this, we first define an $agreeR$ relation between two contexts $\Gamma, \Gamma'$ and a mapping $\xi$ from variables to variables.

\begin{mathpar}
  \inferrule
  { }
  { agreeR\ \xi\ \epsilon\ \epsilon }
  \rname{agreeR-$\epsilon$}

  \inferrule
  { agreeR\ \xi\ \Gamma\ \Gamma' \\
    x \notin FV(\Gamma) \cup FV(\Gamma') }
  { agreeR\ (\xi \cup (x, x))\ (\Gamma, x \utype A) (\Gamma', x \utype A[\xi]) }
  \rname{agreeR-U}

  \inferrule
  { agreeR\ \xi\ \Gamma\ \Gamma' \\
    x \notin FV(\Gamma) \cup FV(\Gamma') }
  { agreeR\ (\xi \cup (x, x))\ (\Gamma, x \ltype A) (\Gamma', x \ltype A[\xi]) }
  \rname{agreeR-L}

  \inferrule
  { agreeR\ \xi\ \Gamma\ \Gamma' \\
    x \notin FV(\Gamma) \cup FV(\Gamma') }
  { agreeR\ \xi\ \Gamma\ (\Gamma', x \utype A) }
  \rname{agreeR-wk}
\end{mathpar}

\subsection{Properties of $agreeR$}\label{agreeRprop}

\begin{lemma}\label{agreerenrefl}
  For any context $\Gamma$ and the identity map $id$ from variables to variables, $agreeR\ id\ \Gamma\ \Gamma$ is always true.
\end{lemma}
\begin{proof}
  By induction on the structure of $\Gamma$ and the definition of $agreeR$.
\end{proof}

\begin{lemma}\label{agreerenpure}
  For contexts $\Gamma, \Gamma'$ and mapping $\xi$, if there is $agreeR\ \xi\ \Gamma\ \Gamma'$ and $\pure{\Gamma}$, then there is $\pure{\Gamma'}$.
\end{lemma}
\begin{proof}
  By induction on the derivation of $agreeR\ \xi\ \Gamma\ \Gamma'$.
\end{proof}

\begin{lemma}\label{agreerenrere}
  For contexts $\Gamma, \Gamma'$ and mapping $\xi$, if there is $agreeR\ \xi\ \Gamma\ \Gamma'$, then there is $agreeR\ \xi\ \pure{\Gamma}\ \pure{\Gamma'}$.
\end{lemma}
\begin{proof}
  By induction on the derivation of $agreeR\ \xi\ \Gamma\ \Gamma'$.
\end{proof}

\subsection{Weakening Theorem}

\begin{lemma}\label{mergeagreereninv}
  For contexts $\Gamma, \Gamma', \Gamma_1, \Gamma_2$ and mapping $\xi$, if there is $agreeR\ \xi\ \Gamma\ \Gamma'$ and $\mrg{\Gamma_1}{\Gamma_2}{\Gamma}$, then there exists $\Gamma_1', \Gamma_2'$ such that $\mrg{\Gamma_1'}{\Gamma_2'}{\Gamma'}$, and $agreeR\ \xi\ \Gamma_1\ \Gamma_1'$ and $agreeR\ \xi\ \Gamma_2\ \Gamma_2'$.
\end{lemma}
\begin{proof}
  By induction on the derivation of $agreeR\ \xi\ \Gamma\ \Gamma'$ and lemmas in Section \ref{agreeRprop}.
\end{proof}

\begin{lemma}\label{renaming}
  For context $\Gamma, \Gamma'$, terms $m, A$ and mapping $\xi$, if there is $\Gamma \vdash m : A$ and $agreeR\ \xi\ \Gamma\ \Gamma'$, then there is $\Gamma' \vdash m[\xi] : A[\xi]$.
\end{lemma}
\begin{proof}
  By induction on the derivation of $\Gamma \vdash m : A$. We shall only discuss the application case in detail, as the other cases are proven by application of the induction hypothesis and the lemmas in Section \ref{agreeRprop}.
  \begin{itemize}
    \item For the \rname{App-U$\rightarrow$} case, Lemma \ref{mergeagreereninv} is applied to split the context $\Gamma$ into two contexts $\Gamma_1'$ and $\Gamma_2'$ such that there is $\mrg{\Gamma_1'}{\Gamma_2'}{\Gamma'}$ and $agreeR\ \xi\ \Gamma_1\ \Gamma_1'$ and $agreeR\ \xi\ \Gamma_2\ \Gamma_2'$. From $\pure{\Gamma_2}$ and Lemma \ref{agreerenpure} we know that there is $\pure{\Gamma_2'}$. At this point, the induction hypothesis allows us to apply \rname{App-U$\rightarrow$} to prove the goal.
    \item For the \rname{App-L$\rightarrow$} case, Lemma \ref{mergeagreereninv} is applied to split the context $\Gamma$ into two contexts $\Gamma_1'$ and $\Gamma_2'$ such that there is $\mrg{\Gamma_1'}{\Gamma_2'}{\Gamma'}$ and $agreeR\ \xi\ \Gamma_1\ \Gamma_1'$ and $agreeR\ \xi\ \Gamma_2\ \Gamma_2'$. At this point, the induction hypothesis allows us to apply \rname{App-L$\rightarrow$} to prove the goal.
    \item For the \rname{App-U$\multimap$} case, Lemma \ref{mergeagreereninv} is applied to split the context $\Gamma$ into two contexts $\Gamma_1'$ and $\Gamma_2'$ such that there is $\mrg{\Gamma_1'}{\Gamma_2'}{\Gamma'}$ and $agreeR\ \xi\ \Gamma_1\ \Gamma_1'$ and $agreeR\ \xi\ \Gamma_2\ \Gamma_2'$. From $\pure{\Gamma_2}$ and Lemma \ref{agreerenpure} we know that there is $\pure{\Gamma_2'}$. At this point, the induction hypothesis allows us to apply \rname{App-U$\multimap$} to prove the goal.
    \item For the \rname{App-L$\multimap$} case, Lemma \ref{mergeagreereninv} is applied to split the context $\Gamma$ into two contexts $\Gamma_1'$ and $\Gamma_2'$ such that there is $\mrg{\Gamma_1'}{\Gamma_2'}{\Gamma'}$ and $agreeR\ \xi\ \Gamma_1\ \Gamma_1'$ and $agreeR\ \xi\ \Gamma_2\ \Gamma_2'$. At this point, the induction hypothesis allows us to apply \rname{App-L$\multimap$} to prove the goal.
  \end{itemize}
\end{proof}

\begin{theorem}
  Weakening is admissible for CLC variables of non-linear type. For context $\Gamma$ and terms $m, A, B$, if there is $\Gamma \vdash m : A$, then there is $\Gamma, x \utype B \vdash m : A$.
\end{theorem}
\begin{proof}
  Using \rname{agreeR-wk} and Lemma \ref{agreerenrefl} a proof of $agreeR\ id\ \Gamma\ (\Gamma, x \utype B)$ can be constructed. Then by Lemma \ref{renaming}, the theorem can be proven.
\end{proof}

\section{Substitution Lemmas of CLC \texttt{(clc_substitution.v)}}

Similar to the proof of weakening, we first define an $agreeS$ relation between two contexts $\Gamma, \Delta$ and a mapping $\sigma$ from variables to terms.

\begin{mathpar}
  \inferrule
  { }
  { agreeS\ \sigma\ \epsilon\ \epsilon }
  \rname{agreeS-$\epsilon$}

  \inferrule
  { agreeS\ \sigma\ \Delta\ \Gamma \\
    x \notin FV(\Delta) \cup FV(\Gamma) }
  { agreeS\ (\sigma \cup (x, x))\ (\Delta, x \utype A[\sigma])\ (\Gamma, x \utype A) }
  \rname{agreeS-U}

  \inferrule
  { agreeS\ \sigma\ \Delta\ \Gamma \\
    x \notin FV(\Delta) \cup FV(\Gamma) }
  { agreeS\ (\sigma \cup (x, x))\ (\Delta, x \ltype A[\sigma])\ (\Gamma, x \ltype A) }
  \rname{agreeS-L}

  \inferrule
  { agreeS\ \sigma\ \Delta\ \Gamma \\
    \overline{\Delta} \vdash n : A[\sigma] \\
    x \notin FV(\Delta) \cup FV(\Gamma) }
  { agreeS\ (\sigma \cup (x, n))\ \Delta\ (\Gamma, x \utype A) }
  \rname{agreeS-wkU}

  \inferrule
  { \mrg{\Delta_1}{\Delta_2}{\Delta} \\
    agreeS\ \sigma\ \Delta_1\ \Gamma \\
    \Delta_2 \vdash n : A[\sigma] \\
    x \notin FV(\Delta) \cup FV(\Gamma) }
  { agreeS\ (\sigma \cup (x, n))\ \Delta\ (\Gamma, x \ltype A) }
  \rname{agreeS-wkL}
\end{mathpar}

\begin{mathpar}
  \inferrule
  { A \preceq B \\
    \overline{\Delta} \vdash B[\sigma] : U_i \\
    agreeS\ \sigma\ \Delta\ (\Gamma, x \utype A) }
  { agreeS\ \sigma\ \Delta\ (\Gamma, x \utype B) }
  \rname{agreeS-convU}

  \inferrule
  { A \preceq B \\
    \overline{\Delta} \vdash B[\sigma] : L_i \\
    \overline{\Gamma} \vdash B : L_i \\
    agreeS\ \sigma\ \Delta\ (\Gamma, x \ltype A) }
  { agreeS\ \sigma\ \Delta\ (\Gamma, x \ltype B) }
  \rname{agreeS-convL}
\end{mathpar}

\subsection{Properties of $agreeS$}\label{agreeSprop}

\begin{lemma}\label{agreesubstrefl}
  For any context $\Gamma$ and identity mapping id, there is $agreeS\ id\ \Gamma\ \Gamma$.
\end{lemma}
\begin{proof}
  By induction on the structure of $\Gamma$.
\end{proof}

\begin{lemma}\label{agreesubstrere}
  For contexts $\Delta, \Gamma$ and mapping $\sigma$, if there is $agreeS\ \sigma\ \Delta\ \Gamma$, then there is $agreeS\ \sigma\ \overline{\Delta}\ \overline{\Gamma}$.
\end{lemma}
\begin{proof}
  By induction on the derivation of $agreeS\ \sigma\ \Delta\ \Gamma$.
\end{proof}

\subsection{Substitution Lemma}

\begin{lemma}\label{mergeagreesubstinv}
  For contexts $\Delta, \Gamma, \Gamma_1, \Gamma_2$ and mapping $\sigma$, if there is $agreeS\ \sigma\ \Delta\ \Gamma$ and $\mrg{\Gamma_1}{\Gamma_2}{\Gamma}$, then there exists contexts $\Delta_1, \Delta_2$ such that $\mrg{\Delta_1}{\Delta_2}{\Delta}$ and $agreeS\ \sigma\ \Delta_1\ \Gamma_1$ and $agreeS\ \sigma\ \Delta_2\ \Gamma_2$.
\end{lemma}
\begin{proof}
  By induction on the derivation of $agreeS\ \sigma\ \Delta\ \Gamma$ and lemmas in Section \ref{agreeSprop}.
\end{proof}

\begin{lemma}
  Generalized Substitution Lemma. For context $\Gamma, \Delta$, terms $m, A$ and mapping $\sigma$, if there is $\Gamma \vdash m : A$ and $agreeS\ \sigma\ \Delta\ \Gamma$, then there is $\Delta \vdash m[\sigma] : A[\sigma]$.
\end{lemma}
\begin{proof}
  The proof proceeds by induction on the derivation of $\Gamma \vdash m : A$. Similar to the proof of Lemma \ref{renaming}, the interesting cases are the application cases where Lemma \ref{mergeagreesubstinv} must be utilized to split the $\mrg{\Gamma_1}{\Gamma_2}{\Gamma}$ judgments for use in the induction hypothesis.
\end{proof}

\subsection{Corollaries of Substitution}

\begin{corollary}\label{substitutionu}
  For contexts $\Gamma_1, \Gamma_2, \Gamma$ and terms $A, B, m, n$, if there is $\Gamma_1, x \utype A \vdash m : B$ and $\pure{\Gamma_2}$ and $\mrg{\Gamma_1}{\Gamma_2}{\Gamma}$ and $\Gamma_2 \vdash n : A$, then there is $\Gamma \vdash m[n/x]: B[n/x]$.
\end{corollary}

\begin{corollary}\label{substitutionl}
  For contexts $\Gamma_1, \Gamma_2, \Gamma$ and terms $A, B, m, n$, if there is $\Gamma_1, x \ltype A \vdash m : B$ and $\mrg{\Gamma_1}{\Gamma_2}{\Gamma}$ and $\Gamma_2 \vdash n : A$, then there is $\Gamma \vdash m[n/x]: B[n/x]$.
\end{corollary}

\begin{corollary}\label{contextconvu}
  For context $\Gamma$, terms $m, A, B, C$ and natural number $i$, if there is $B \equiv A$ and $\overline{\Gamma} \vdash A : U_i$ and $\Gamma, x \utype A \vdash m : C$, then there is $\Gamma, x \utype B \vdash m : C$.
\end{corollary}

\begin{corollary}\label{contextconvl}
  For context $\Gamma$, terms $m, A, B, C$ and natural number $i$, if there is $B \equiv A$ and $\overline{\Gamma} \vdash A : L_i$ and $\Gamma, x \ltype A \vdash m : C$, then there is $\Gamma, x \ltype B \vdash m : C$.
\end{corollary}

\section{Typing Validity of CLC \texttt{(clc_validity.v)}}

In this section, we prove that the types of all CLC terms are themselves well-sorted.

\begin{lemma}\label{mergecontextokinv}
  For contexts $\Gamma_1, \Gamma_2, \Gamma$, if there is $\mrg{\Gamma_1}{\Gamma_2}{\Gamma}$ and $\Gamma \vdash$, then there is $\Gamma_1 \vdash$ and $\Gamma_2 \vdash$.
\end{lemma}
\begin{proof}
  By induction on the derivation of $\mrg{\Gamma_1}{\Gamma_2}{\Gamma}$ and the properties of $\mrg{\_}{\_}{\_}$ discussed in Section \ref{mergeprop}.
\end{proof}

\begin{theorem}\label{validity}
  The validity of typing theorem. For any context $\Gamma$ and terms $m, A$, if there is $\Gamma \vdash$ and $\Gamma \vdash m : A$, then there exists sort $s$ and natural number $i$ such that $\overline{\Gamma} \vdash A : s_i$.
\end{theorem}

\section{Subject Reduction of CLC \texttt{(clc_soundness.v)}}

\begin{theorem}
  For any context $\Gamma$ and terms $m, n, A$, if $\Gamma \vdash$ and $\Gamma \vdash m : A$ and $m \step n$, then there is $\Gamma \vdash n : A$.
\end{theorem}
\begin{proof}
  The proof proceeds by induction on the derivation of $\Gamma \vdash m : A$. The interesting cases are the application cases which we shall discuss in detail.
  \begin{itemize}
    \item For the \rname{App-U$\rightarrow$} case,
  \end{itemize}
\end{proof}

\end{document}