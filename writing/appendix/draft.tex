\documentclass{article}

\usepackage[utf8]{inputenc}
\usepackage[margin=1in]{geometry}

% packages
\usepackage{graphicx}
\usepackage{amsthm}
\usepackage{amsmath}
\usepackage{amssymb}
\usepackage{mdwtab}
\usepackage{syntax}
\usepackage{stmaryrd}
\usepackage{mathtools}
\usepackage{mathpartir}
\usepackage{listings}
\usepackage{float}
\usepackage{tikz-cd}
\usepackage{enumitem}
\usepackage{minted}

\setlist[itemize]{leftmargin=1.6em}
\renewcommand{\syntleft}{}
\renewcommand{\syntright}{}
\setlength{\grammarparsep}{7pt}
\setlength{\grammarindent}{7em}
\newcommand{\indalt}[1][2]{\\\hspace*{-0.2em}\textbar\quad}
\newcommand{\rname}[1]{\textsc{\footnotesize #1}}
\newcommand{\pure}[1]{|#1|}
\newcommand{\refl}{\text{refl}}
\newcommand{\letin}[3]{\ $\text{let }#1\text{ := }#2\text{ in }#3$\ }
\newcommand{\ind}[1]{\text{Ind}_{#1}}
\newcommand{\constr}{\text{Constr}}
\newcommand{\case}{\text{Case}}
\newcommand{\dcase}{\text{DCase}}
\newcommand{\fix}{\text{Fix }}
\newcommand{\sep}{\text{ | }}
\newcommand{\unit}{\text{unit}}
\newcommand{\new}{\text{new}}
\newcommand{\free}{\text{free}}
\newcommand{\get}{\text{get}}
\newcommand{\set}{\text{set}}
\newcommand{\session}{\text{session}}
\newcommand{\channel}{\text{channel}}
\newcommand{\open}{\text{open}}
\newcommand{\close}{\text{close}}
\newcommand{\send}{\text{send}}
\newcommand{\recv}{\text{recv}}
\newcommand{\SEND}{\texttt{SEND}}
\newcommand{\RECV}{\texttt{RECV}}
\newcommand{\END}{\texttt{END}}
\newcommand{\utype}{:_{\scriptscriptstyle U}}
\newcommand{\ltype}{:_{\scriptscriptstyle L}}
\newcommand{\stype}[1]{:_#1}
\newcommand{\step}{\leadsto}
\newcommand{\pstep}{\leadsto_p}
\newcommand{\mrg}[3]{#1\ddagger#2\ddagger#3}
\newcommand{\erase}[1]{\llbracket #1 \rrbracket}
\newcommand{\lift}[1]{\llparenthesis #1 \rrparenthesis}
\newcommand{\lrangle}[1]{\langle #1 \rangle}
\newcommand{\ucons}{constructor_{\scriptscriptstyle U}}
\newcommand{\lcons}{constructor_{\scriptscriptstyle L}}
\newcommand{\scons}{constructor_{s}}
\newcommand{\inl}{\text{inl}}
\newcommand{\inr}{\text{inr}}


\title{The Calculus of Linear Constructions --- Technical Report}
\author{Qiancheng Fu}

\begin{document}
\maketitle
\tableofcontents

\section{Introduction}
This extended report is meant to accompany our paper of the same title. Here, we describe the meta-theory of CILC and their proofs in detail. All the results presented here have been formalized and proven correct in the Coq Proof Assistant.

\section{Syntax of CLC \texttt{(clc_ast.v)}}
\begin{figure}[H]
  \begin{grammar}
    <$i$> := 0 | 1 | 2 ... \phantom{* |} \hspace*{2.4em} universe levels

    <$s, t$> ::= $U$ | $L$ \phantom{| $x$} \hspace*{4.6em} sorts

    <$m, n, A, B, M$> ::= $U_i$ | $L_i$ | $x$ \hspace*{3em} expressions
    \indalt $(x :_s A) \rightarrow B$
    \indalt $(x :_s A) \multimap B$
    \indalt $\lambda x :_s A. n$
    \indalt $m\ n$
  \end{grammar}
\end{figure}

\section{Reduction and Equality of CLC \texttt{(clc_ast.v)}}
\begin{figure}[H]
  \begin{mathpar}
    \inferrule
    { m_1 \step^* n \\ m_2 \step^* n }
    { m_1 \equiv m_2 : A }
    \rname{Join}

    \inferrule
    {  }
    { (\lambda x \stype{s}A.m)\ n \step m[n/x] }
    \rname{Step-$\beta$}

    \inferrule
    { A \step A' }
    { \lambda x \stype{s}A.m \step \lambda x \stype{s}A' .m }
    \rname{Step-$\lambda$L}

    \inferrule
    { m \step m' }
    { \lambda x \stype{s}A.m \step \lambda x \stype{s}A.m' }
    \rname{Step-$\lambda$R}

    \inferrule
    { A \pstep A' }
    { (x \stype{s} A) \rightarrow B \step (x \stype{s} A') \rightarrow B }
    \rname{Step-L$\rightarrow$}

    \inferrule
    { B \pstep B' }
    { (x \stype{s} A) \rightarrow B \step (x \stype{s} A) \rightarrow B' }
    \rname{Step-R$\rightarrow$}

    \inferrule
    { A \pstep A' }
    { (x \stype{s} A) \multimap B \step (x \stype{s} A') \multimap B }
    \rname{Step-L$\multimap$}

    \inferrule
    { B \pstep B' }
    { (x \stype{s} A) \multimap B \step (x \stype{s} A) \multimap B' }
    \rname{Step-R$\multimap$}

    \inferrule
    { m \step m' }
    { m\ n \step m'\ n }
    \rname{Step-AppL}

    \inferrule
    { n \step n' }
    { m\ n \step m\ n' }
    \rname{Step-AppR}
  \end{mathpar}
  \label{red}
\end{figure}

\section{Confluence of CLC \texttt{(clc_confluence.v)}}

\subsection{Parallel Reduction}
To prove the confluence property of CLC, we employ the standard technique utilizing parallel reductions.

\begin{figure}[H]
  \begin{mathpar}
    \inferrule
    { }
    { x \pstep x }
    \rname{Pstep-Var}

    \inferrule
    { }
    { s_i \pstep s_i }
    \rname{Pstep-Sort}

    \inferrule
    { m \pstep m' }
    { }
  \end{mathpar}
  \label{pred}
\end{figure}

\end{document}